\documentclass{book}

\usepackage{etoolbox}

\makeatletter
\def\subtitle#1{\gdef\@subtitle{#1}}
\patchcmd\maketitle
  {{\LARGE \@title \par}}
  {{\LARGE \@title \par}%
   \vskip 1.5em
   {\Large \@subtitle \par}}
\makeatother

\usepackage[utf8x]{inputenc}
\usepackage{amsmath,mathtools,bbm}
\usepackage{amsfonts}
\usepackage{amssymb}
\usepackage{graphicx}
\usepackage{listings}
\usepackage{appendix}
\usepackage{enumitem}
\usepackage{bm}
\usepackage{multicol}
\usepackage{geometry}
\usepackage{colortbl}
\usepackage{changepage}
\usepackage{color}
\usepackage{mathrsfs}
\usepackage{bigints}
\usepackage{pdflscape}
\usepackage{adjustbox}
\usepackage{tocloft}
\usepackage{lscape}
\usepackage[colorlinks=true,
            linkcolor=blue,
            urlcolor=blue,
            citecolor=blue]{hyperref}

\definecolor{codegreen}{rgb}{0,0.6,0}
\definecolor{codegray}{rgb}{0.5,0.5,0.5}
\definecolor{codepurple}{rgb}{0.58,0,0.82}
\definecolor{backcolour}{rgb}{0.97,0.97,0.97}
\definecolor{stringorange}{rgb}{0.95, 0.4, 0.18}

\lstdefinestyle{mystyle}{
    backgroundcolor=\color{backcolour},   
    commentstyle=\color{codegreen},
    keywordstyle=\color{codepurple},
    numberstyle=\tiny\color{codegray},
    stringstyle=\color{stringorange},
    basicstyle=\ttfamily\footnotesize,
    breakatwhitespace=false,         
    breaklines=true,                 
    captionpos=b,                    
    keepspaces=true,                 
    numbers=left,                    
    numbersep=5pt,                  
    showspaces=false,                
    showstringspaces=false,
    showtabs=false,                  
    tabsize=2
}
\lstset{style=mystyle}

\sloppy
\usepackage{tikz, lipsum}% http://ctan.org/pkg/{pgf,lipsum}
\newcommand*{\chapnumfont}{\normalfont\sffamily\huge\bfseries}
\newcommand*{\printchapternum}{
  \begin{tikzpicture}
    \draw[fill,color=gray75] (0,0) rectangle (2cm,2cm);
    \draw[color=white] (1cm,1cm) node { \chapnumfont\thechapter };
  \end{tikzpicture}
}
\newcommand*{\chaptitlefont}{\normalfont\sffamily\Huge\bfseries}
\newcommand*{\printchaptertitle}[1]{\flushright\chaptitlefont#1}

\makeatletter
% \@makechapterhead prints regular chapter heading.
% Taken directly from report.cls and modified.
\def\@makechapterhead#1{%
  \vspace*{50\p@}%
  {\parindent \z@ \raggedleft
    \ifnum \c@secnumdepth >\m@ne
        \printchapternum
        \par\nobreak
        \vskip 20\p@
    \fi
    \interlinepenalty\@M
    \printchaptertitle{#1}\par\nobreak
    \vskip 40\p@
  }}
% \@makeschapterhead prints starred chapter heading.
% Taken directly from report.cls and modified.
\def\@makeschapterhead#1{%
  \vspace*{50\p@}%
  {\parindent \z@ \raggedleft
    \interlinepenalty\@M
    \printchaptertitle{#1}\par\nobreak
    \vskip 40\p@
  }}
\makeatother

\makeatletter\@addtoreset{chapter}{part}\makeatother%

\renewcommand{\cfttoctitlefont}{\hfill\Huge\bfseries\sffamily}

\renewcommand\contentsname{Table of contents}

\definecolor{gray75}{gray}{0.75}

\title{\Huge \bfseries\sffamily Diffusing Ideas}
\subtitle{\Large \bfseries\sffamily \color{gray75} Software, noise and building mathematical toys}
\author{\bfseries\sffamily Robert J. Hardwick  \and C. M. Gomez-Perales}
\date{\today}

\begin{document}
\begin{titlepage}
\centering
\vspace*{1.5\baselineskip}
{\color{gray75}\rule{13cm}{1.3pt}}\vspace*{-\baselineskip}\vspace*{2pt} % Thick horizontal rule
{\color{gray75}\rule{13cm}{0.4pt}} \\ % Thin horizontal rule
\vspace{1.2\baselineskip} % Whitespace 
{\Huge \bfseries\sffamily Diffusing Ideas} \\ 
\vspace{4mm}
{\Large \bfseries\sffamily \color{gray75} Software, noise and building mathematical toys \\}
\vspace*{0.75\baselineskip}
{\color{gray75}\rule{13cm}{0.4pt}}\vspace*{-\baselineskip}\vspace*{2.75pt} % Thick horizontal rule
{\color{gray75}\rule{13cm}{1.3pt}} \\ % Thin horizontal rule
\vspace{1.0\baselineskip} % Whitespace 
{\large \bfseries\sffamily Robert J. Hardwick and C. M. Gomez-Perales \\
\vspace*{1.2\baselineskip}}
\today
\vfill
Shared by the authors under an \href{https://opensource.org/licenses/MIT}{MIT License}.\\ \vspace{1mm}
The code to compile this book is open source and can be found in this repository: \url{https://github.com/umbralcalc/diffusing-ideas}.
\end{titlepage}

\chapter*{Introduction}

\emph{Diffusing Ideas} is a book of research exploration and software development which we have written for the interest of mathematically-inclined programmers and computational scientists. It's the output of many interrelated projects over several years which have sought to generalise the computational mathematics of simulating, statistically inferring, manipulating and automatically controlling stochastic phenomena as far as possible.

The book accompanies a lot of new open-source scientific software written predominantly in Go~\cite{golang}, and with some Python~\cite{pythonlang} to top it off. A major motivation for creating these new tools is to prepare a foundation of code from which to develop new and more complex applications. We also hope that the resulting framework will enable anyone to explore and study new phenomena effectively, regardless of their scientific background.

The need to properly test all this software has also provided a wonderful excuse to study and play with an extensive range of mathematical toy models. we've chosen these models based on a fairly broad background of interests, but also to illustrate the remarkable cross-disciplinary applicability of stochastic processes. However, we've often found that mathematical formalities can obscure the computations that a programmer must implement. So, while we've tried to be as ambitious as possible with the level of technical sophistication in these models, we've also tried to write the expressions in a computer-friendly way where feasible\footnote{For example, we'll typically be thinking more in terms of `matrices' and less about `operators'.} and have added code snippets throughout.

A quick note on the code: any software that we describe in this book (including the software which compiles the book itself~\cite{diffusingideasbookgithub}) will always be shared under a MIT License~\cite{mitlicense} in a public Git repository.\footnote{The repositories will always be somewhere on this list: \href{https://github.com/umbralcalc?tab=repositories}{https://github.com/umbralcalc?tab=repositories}.} Forking these repositories and submitting pull requests for new features or applications is strongly encouraged too, though we apologise in advance if we don't follow these up very quickly as all of this work has to be conducted independently in free time, outside of work hours.

No quest would be complete without a guide, so we think this introduction should end with a list of the key milestones in the book; comprising its four major parts. These parts each correspond to answering one of the following interdependent research questions:

\begin{enumerate}[leftmargin=2.5\parindent] 
\item[{\bfseries\sffamily Part 1.}]{How do we simulate a general set of stochastic phenomena?}
\item[{\bfseries\sffamily Part 2.}]{How do we then learn/identify the answer to {\bfseries\sffamily Part 1} from real-world data?}
\item[{\bfseries\sffamily Part 3.}]{How do we simulate a general set of control policies to interact with the answer to {\bfseries\sffamily Part 1}?}
\item[{\bfseries\sffamily Part 4.}]{How do we then optimise the answer to {\bfseries\sffamily Part 3} to achieve a specified control objective?} 
\end{enumerate}

\newpage \ \newpage
{\sffamily \tableofcontents}
\mainmatter

\part*{
\includegraphics*[width=14cm]{images/page-design-1.png}}

\chapter{\sffamily Building a generalised simulator}

{\bfseries\sffamily Concept.} To design and build a generalised simulation engine that is able to generate samples from practically any real-world stochastic processes that a researcher could encounter. With such a thing pre-built and self-contained, it can become the basis upon which to build generalised software solutions for a lot of different problems. For the mathematically-inclined, this chapter will require the introduction of a new formalism which we shall refer back to throughout the book. For the programmers, the public Git repository for the code that is described in this chapter can be found here: \href{https://github.com/umbralcalc/stochadex}{https://github.com/umbralcalc/stochadex}.

\section{\sffamily Computational formalism}

Before we dive into software design we need to mathematically define the general computational approach that we're going to take. Because the language of stochastic processes is primarily mathematics, we'd argue this step is essential in enabling a really general description. From experience, it seems reasonable to start by writing down the following formula which describes iterating some arbitrary process forward in time (by one finite step) and adding a new row each to some matrix $X_{0:{\sf t}} \rightarrow X_{0:{\sf t}+1}$
%%
\begin{align}
X^{i}_{{\sf t}+1} &= F^{i}_{{\sf t}+1}(X_{0:{\sf t}},z,{\sf t}) \,, \label{eq:x-step-def}
\end{align}
%%
where: $i$ is an index for the dimensions of the `state' space; ${\sf t}$ is the current time index for either a discrete-time process or some discrete approximation to a continuous-time process; $X_{0:{\sf t}+1}$ is the next version of $X_{0:{\sf t}}$ after one timestep (and hence one new row has been added); $z$ is a vector of arbitrary size which contains the `hidden' other parameters that are necessary to iterate the process; and $F^i_{{\sf t}+1}(X_{0:{\sf t}},z,{\sf t})$ as the latest element of an arbitrary matrix-valued function. 

Throughout the book, the notation $A_{{{\sf b}:{\sf c}}}$ will always refer to a slice of rows from index ${\sf b}$ to ${\sf c}$ in a matrix (or row vector) $A$. As we shall discuss shortly, $F^i_{{\sf t}+1}(X_{0:{\sf t}},z,{\sf t})$ may represent not just operations on deterministic variables, but also on stochastic ones. There is also no requirement for the function to be continuous.

\begin{figure}[h]
\centering
\includegraphics[width=10cm]{images/chapter-1-fundamental-loop.drawio.png}
\caption{Graph representation of Eq.~(\ref{eq:x-step-def}).}
\label{fig:fundamental-loop}
\end{figure}

The basic computational idea here is illustrated in Fig.~\ref{fig:fundamental-loop}; we iterate the matrix $X$ forward in time by a row, and use its previous version $X_{0:{\sf t}}$ as an entire matrix input into a function which populates the elements of its latest rows. In pseudocode you could easily write something with the same idea in it, and it would probably look something like the method diagram in Fig.~\ref{fig:fundamental-loop-code}.

\begin{figure}[h]
\centering
\includegraphics[width=10cm]{images/chapter-1-fundamental-loop-code.drawio.png}
\caption{Pseudocode representation of Eq.~(\ref{eq:x-step-def}).}
\label{fig:fundamental-loop-code}
\end{figure}

Pretty simple! But why go to all this trouble of storing matrix inputs for previous values of the same process? It's true that this is mostly redundant for \emph{Markovian} phenomena, i.e., processes where their only memory of their history is the most recent value they took. However, for a large class of stochastic processes a full memory\footnote{Or memory at least within some window.} of past values is essential to consistently construct the sample paths moving forward. This is true in particular for \emph{non-Markovian} phenomena, where the latest values don't just depend on the immediately previous ones but can depend on values which occured much earlier in the process as well.

For more complex physical models and integrators, the distinct notions of `numerical timestep' and `total elapsed continuous time' will crop up quite frequently. Hence, before moving on further details, it will be important to define the total elapsed time variable $t({\sf t})$ for processes which are defined in continuous time. Assuming that we have already defined some function $\delta t({\sf t})$ which returns the specific change in continuous time that corresponds to the step ${\sf t}-1 \rightarrow {\sf t}$, we will always be able to compute the total elapsed time through the relation
%%
\begin{align}
t({\sf t}) &= \sum^{{\sf t}}_{{\sf t}'=0}\delta t({\sf t}') \label{eq:t-steps-sum} \,.
\end{align}
%%
This seems a lot of effort, no? Well it's important to remember that our steps in continuous time may not be constant, so by defining the $\delta t({\sf t})$ function and summing over it we can enable this flexibility in the computation. In case the summation notation is no fun for programmers; we're simply adding up all of the differences in time to get a total. We've illustrated this in Fig.~\ref{fig:time-step-summation} for more clarity.

\begin{figure}[h]
\centering
\includegraphics[width=10cm]{images/chapter-1-time-step-summation.drawio.png}
\caption{Illustration of Eq.~(\ref{eq:t-steps-sum}).}
\label{fig:time-step-summation}
\end{figure}

So, now that we've mathematically defined a really general notion of iterating the stochastic process forward in time, it makes sense to discuss some simple examples. For instance, it is frequently possible to split $F$ up into deteministic (denoted $D$) and stochastic (denoted $S$) matrix-valued functions like so
%%
\begin{align}
& F^{i}_{{\sf t}+1}(X_{0:{\sf t}},z,{\sf t}) = D^{i}_{{\sf t}+1}(X_{0:{\sf t}},z,{\sf t}) + S^{i}_{{\sf t}+1}(X_{0:{\sf t}},z,{\sf t}) \,.
\end{align}
%%
In the case of stochastic processes with continuous sample paths, it's also nearly always the case with mathematical models of real-world systems that the deterministic part will at least contain the term $D^{i}_{{\sf t}+1}(X_{0:{\sf t}},z,{\sf t}) = X^i_{\sf t}$ because the overall system is described by some stochastic differential equation. This is not a really requirement in our general formalism, however.

What about the stochastic term? For example, if we wanted to consider a \emph{Wiener process noise}, we can define $W^i_{{\sf t}}$ is a sample from a Wiener process for each of the state dimensions indexed by $i$ and our formalism becomes
%%
\begin{align}
& S^{i}_{{\sf t}+1}(X_{0:{\sf t}},z,{\sf t}) = W^i_{{\sf t}+1}-W^i_{\sf t} \label{eq:wiener}\,.
\end{align}
%%
One draws the increments $W^i_{{\sf t}+1}-W^i_{\sf t}$ from a normal distribution with a mean of $0$ and a variance equal to the length of continuous time that the step corresponded to $\delta t({\sf t}+1)$, i.e., the probability density $P_{{\sf t}+1}(x^i)$ of the increments $x^i=W^i_{{\sf t}+1}-W^i_{\sf t}$ is
%%
\begin{align}
P_{{\sf t}+1}(x^i) &= {\sf NormalPDF}[x^i;0,\delta t({\sf t}+1)] \,.
\end{align}
%%
Note that for state spaces with dimensions $>1$, we could also allow for non-trivial cross-correlations between the noises in each dimension. In pseudocode, the Wiener process is schematically represented by Fig.~\ref{fig:wiener-process}.

\begin{figure}[h]
\centering
\includegraphics[width=8cm]{images/chapter-1-wiener-process.drawio.png}
\caption{Schematic of code for a Wiener process.}
\label{fig:wiener-process}
\end{figure}

In another example, to model \emph{geometric Brownian motion noise} we would simply have to multiply $X^i_{\sf t}$ to the Wiener process like so
%%
\begin{align}
& S^{i}_{{\sf t}+1}(X_{0:{\sf t}},z,{\sf t}) = X^i_{\sf t}(W^i_{{\sf t}+1}-W^i_{\sf t})\label{eq:gbm} \,.
\end{align}
%%
Here we have implicitly adopted the Itô interpretation to describe this stochastic integration. Given a carefully-defined integration scheme other interpretations of the noise would also be possible with our formalism too, e.g., Stratonovich\footnote{Which would implictly give $S^{i}_{{\sf t}+1}(X_{0:{\sf t}},z,{\sf t}) = (X^i_{{\sf t}+1}+X^i_{\sf t})(W^i_{{\sf t}+1}-W^i_{\sf t}) / 2$ for Eq.~(\ref{eq:gbm}).} or others within the more general `$\alpha$-family'~\cite{van1992stochastic,risken1996fokker,rog-will-2000}. The pseudocode for any of these should hoepfully be fairly straightforward to deduce based on the lines we've already written above.

\begin{figure}[h]
\centering
\includegraphics[width=12cm]{images/chapter-1-ito-lemma.drawio.png}
\caption{Schematic of code for Eq.~(\ref{eq:general-wiener}).}
\label{fig:ito-lemma}
\end{figure}

We can imagine even more general processes that are still Markovian. One example of these in a single-dimension state space would be to define the noise through some general function of the Wiener process like so
%%
\begin{align}
S^0_{{\sf t}+1}(X_{0:{\sf t}},z,{\sf t}) &= g[W^0_{{\sf t}+1},t({\sf t}+1)]-g[W^0_{\sf t}, t({\sf t})] \\
&= \bigg[ \frac{\partial g}{\partial t} + \frac{1}{2}\frac{\partial^2 g}{\partial x^2} \bigg] \delta t ({\sf t}+1) + \frac{\partial g}{\partial x} (W^0_{{\sf t}+1}-W^0_{\sf t}) \label{eq:general-wiener}\,,
\end{align}
%%
where $g(x,t)$ is some continuous function of its arguments which has been expanded out with Itô's Lemma on the second line. Note also that the computations in Eq.~(\ref{eq:general-wiener}) could be performed with numerical derivatives in principle, even if the function were extremely complicated. This is unlikely to be the best way to describe the process of interest, however, the mathematical expressions above can still be made a bit more meaningful to the programmer in this way. The pseudocode in general would look something like Fig.~\ref{fig:ito-lemma}.

Let's now look at a more complicated type of noise. For example, we might consider sampling from a \emph{fractional Brownian motion} process $[B_{H}]_{\sf t}$, where $H$ is known as the `Hurst exponent'. Following Ref.~\cite{decreusefond1999stochastic}, we can simulate this process in one of our state space dimensions by modifying the standard Wiener process by a fairly complicated integral factor which looks like this
%%
\begin{align}
S^{0}_{{\sf t}+1}(X_{0:{\sf t}},z,{\sf t}) &= \frac{(W^0_{{\sf t}+1} - W^0_{\sf t})}{\delta t({\sf t})}\int^{t({\sf t}+1)}_{t({\sf t})}{\rm d}t' \frac{(t'-t)^{H-\frac{1}{2}}}{\Gamma (H+\frac{1}{2})} {}_2F_1 \bigg( H-\frac{1}{2};\frac{1}{2}-H;H+\frac{1}{2};1-\frac{t'}{t}\bigg) \label{eq:fbm} \,,
\end{align}
%%
where $S^{0}_{{\sf t}+1}(X_{0:{\sf t}},z,{\sf t})=[B_{H}]_{{\sf t}+1}-[B_{H}]_{{\sf t}}$. The integral in Eq.~(\ref{eq:fbm}) can be approximated using an appropriate numerical procedure (like the trapezium rule, for instance). In the expression above, we have used the symbols ${}_2F_1$ and $\Gamma$ to denote the ordinary hypergeometric and gamma functions, respectively. A computational form of this integral is illustrated in Fig.~\ref{fig:fractional-brownian-motion} to try and disentangle some of the mathematics as a program.

\begin{figure}[h]
\centering
\includegraphics[width=11cm]{images/chapter-1-fractional-brownian-motion.drawio.png}
\caption{Schematic of code for Eq.~(\ref{eq:fbm}).}
\label{fig:fractional-brownian-motion}
\end{figure}

So far we have mostly been discussing noises with continuous sample paths, but we can easily adapt our computation to discontinuous sample paths as well. For instance, \emph{Poisson process noises} would generally take the form
%%
\begin{align}
S^{i}_{{\sf t}+1}(X_{0:{\sf t}},z,{\sf t}) &= [N_{\lambda}]^i_{{\sf t}+1}-[N_{\lambda}]^i_{\sf t}\,,
\end{align}
%%
where $[N_{\lambda}]^i_{\sf t}$ is a sample from a Poisson process with rate $\lambda$. One can think of this process as counting the number of events which have occured up to the given interval of time, where the intervals between each succesive event are exponentially distributed with mean $1/\lambda$. Such a simple counting process could be simulated exactly by explicitly setting a newly-drawn exponential variate to the next continuous time jump ${\delta t}({\sf t}+1)$ and iterating the counter. Other exact methods exist to handle more complicated processes involving more than one type of `event', such as the Gillespie algorithm~\cite{gillespie1977exact} --- though these techniques are not always be applicable in every situation.

Is using step size variation always possible? If we consider a \emph{time-inhomogeneous Poisson process noise}, which would generally take the form
%%
\begin{align}
S^{i}_{{\sf t}+1}(X_{0:{\sf t}},z,{\sf t}) &= [N_{\lambda ({\sf t}+1)}]^i_{{\sf t}+1}-[N_{\lambda ({\sf t})}]^i_{\sf t}\,,
\end{align}
%%
the rate $\lambda ({\sf t})$ has become a deterministically-varying function in time. In this instance, it likely not be accurate to simulate this process by drawing exponential intervals with a mean of $1/\lambda ({\sf t})$ because this mean could have changed by the end of the interval which was drawn. An alternative approach (which is more generally capable of simulating jump processes but is an approximation) first uses a small time interval $\tau$ such that the most likely thing to happen in this period is nothing, and then the probability of the event occuring is simply given by
%%
\begin{align}
p({\sf event}) &= \frac{\lambda ({\sf t})}{\lambda ({\sf t}) + \frac{1}{\tau}} \label{eq:rejection}\,.
\end{align}
%%
This idea can be applied to phenomena with an arbitrary number of events and works well as a generalised approach to event-based simulation, though its main limitation is worth remembering; in order to make the approximation good, $\tau$ often must be quite small and hence our simulator must churn through a lot of steps. From now on we'll refer to this well-known technique as the \emph{rejection method}. Fig.~\ref{fig:inhomogeneous-poisson} may also help to understand this concept from the programmer's perspective.

\begin{figure}[h]
\centering
\includegraphics[width=9cm]{images/chapter-1-inhomogeneous-poisson.drawio.png}
\caption{Schematic of code for an inhomogeneous Poisson process.}
\label{fig:inhomogeneous-poisson}
\end{figure}

There are a few extensions to the simple Poisson process that introduce additional stochastic processes. \emph{Cox (doubly-stochastic) processes}, for instance, are basically where we replace the time-dependent rate $\lambda ({\sf t})$ with independent samples from some other stochastic process $\Lambda ({\sf t})$. For example, a Neyman-Scott process~\cite{neyman1958statistical} can be mapped as a special case of this because it uses a Poisson process on top of another Poisson process to create maps of spatially-distributed points. In our formalism, a two-state implementation of the Cox process noise would look like
%%
\begin{align}
S^{0}_{{\sf t}+1}(X_{0:{\sf t}},z,{\sf t}) &= \Lambda ({\sf t}+1) \\
S^{1}_{{\sf t}+1}(X_{0:{\sf t}},z,{\sf t}) &= [N_{S^{0}_{{\sf t}+1}}]^i_{{\sf t}+1}-[N_{S^{0}_{{\sf t}}}]^i_{\sf t}\,.
\end{align}
%%
This process could be simulated using the pseudocode we wrote for the time-inhomogeneous Poisson process previously --- where we would just replace \texttt{EventRateLambdaFunction} with a function that generates the stochastic rate $\Lambda ({\sf t})$.

Another extension is \emph{compound Poisson process noise}, where it's the count values $[N_{\lambda}]^i_{\sf t}$ which are replaced by independent samples $[J_{\lambda}]^i_{\sf t}$ from another probability distribution, i.e.,
%%
\begin{align}
S^{i}_{{\sf t}+1}(X_{0:{\sf t}},z,{\sf t}) &= [J_{\lambda}]^i_{{\sf t}+1}-[J_{\lambda}]^i_{\sf t}\,.
\end{align}
%%
Note that the rejection method of Eq.~(\ref{eq:rejection}) can be employed effectively to simulate any of these extensions as long as a sufficiently small $\tau$ is chosen. Once again, the pseudocode we wrote previously would be sufficient to simulate this process with one tweak: add into the \texttt{DrawNewEventIncrement} function the calling of a function which generates the $[J_{\lambda}]^i_{\sf t}$ samples and output these if the event occurs.

All of the examples we have discussed so far are Markovian. Given that we have explicitly constructed the formalism to handle non-Markovian phenomena as well, it would be worthwhile going some examples of this kind of process too. \emph{Self-exciting process noises} would generally take the form
%%
\begin{align}
S^{0}_{{\sf t}+1}(X_{0:{\sf t}},z,{\sf t}) &= {\cal I}_{{\sf t}+1} (X_{0:{\sf t}},z,{\sf t}) \\
S^{1}_{{\sf t}+1}(X_{0:{\sf t}},z,{\sf t}) &= [N_{S^{0}_{{\sf t}+1}}]^i_{{\sf t}+1}-[N_{S^{0}_{{\sf t}}}]^i_{\sf t} \,,
\end{align}
%%
where the stochastic rate ${\cal I}_{{\sf t}+1} (X_{0:{\sf t}},z,{\sf t})$ now depends on the history explicitly. Amongst other potential inputs we can see, e.g., Hawkes processes~\cite{hawkes1971spectra} as an example of above by substituting 
%%
\begin{align}
{\cal I}_{{\sf t}+1} (X_{0:{\sf t}},z,{\sf t}) &= \mu + \sum^{{\sf t}}_{{\sf t}'=0}\gamma [t({\sf t})-t({\sf t}')]S^{1}_{{\sf t}'} \,,
\end{align}
%%
where $\gamma$ is the `exciting kernel' and $\mu$ is some constant background rate. In order to simulate a Hawkes process using our formalism, the pseudocode would be something like Fig.~\ref{fig:hawkes-process}.

\begin{figure}[h]
\centering
\includegraphics[width=9cm]{images/chapter-1-hawkes-process.drawio.png}
\caption{Schematic of code for a Hawkes process.}
\label{fig:hawkes-process}
\end{figure}

Note that this idea of integration kernels could also be applied back to our Wiener process. For example, another type of non-Markovian phenomenon that frequently arises across physical and life systems integrates the Wiener process history like so
%%
\begin{align}
S^{0}_{{\sf t}+1}(X_{0:{\sf t}},z,{\sf t}) &= W^0_{{\sf t}+1}-W^0_{\sf t}\\
S^{1}_{{\sf t}+1}(X_{0:{\sf t}},z,{\sf t}) &= u\sum^{{\sf t}}_{{\sf t}'=0}e^{-u[t({\sf t})-t({\sf t}')]} S^{0}_{{\sf t}'}\,,
\end{align}
%%
where $u$ is inversely proportional to the length of memory in continuous time.

\section{\sffamily Software design}

So we've proposed a computational formalism and then studied it in more detail to demonstrate that it can cope with a variety of different stochastic phenomena. Now we're ready to summarise what we want the stochadex software package to be able to do. But what's so complicated about Eq.~(\ref{eq:x-step-def})? Can't we just implement an iterative algorithm with a single function? It's true that the fundamental concept is very straightforward, but as we'll discuss in due course; the stochadex needs to have a lot of configurable features so that it's applicable in different situations. Ideally, the stochadex sampler should be designed to try and maintain a balance between performance and flexibility of utilisation.

If we begin with the obvious first set of criteria; we want to be able to freely configure the iteration function $F$ of Eq.~(\ref{eq:x-step-def}) and the timestep function $t$ of Eq.~(\ref{eq:t-steps-sum}) so that any process we want can be described. The point at which a simulation stops can also depend on some algorithm termination condition which the user should be able to specify up-front.

\begin{figure}[h]
\centering
\includegraphics[width=13cm]{images/chapter-1-stochadex-data-types.drawio.png}
\caption{A relational summary of the configuration data types in the stochadex.}
\label{fig:data-types-design}
\end{figure}

Once the user has written the code to create these functions for the stochadex, we want to then be able to recall them in future only with configuration files while maintaining the possibility of changing their simulation run parameters. This flexibility should facilitate our uses for the simulation later in the book, and from this perspective it also makes sense that the parameters should include the random seed and initial state value.

The state history matrix $X$ should be configurable in terms of its number of rows --- what we'll call the `state width' --- and its number of columns --- what we'll call the `state history depth'. If we were to keep increasing the state width up to millions of elements or more, it's likely that on most machines the algorithm performance would grind to a halt when trying to iterate over the resulting $X$ within a single thread. Hence, before the algorithm or its performance in any more detail, we can pre-empt the requirement that $X$ should represented in computer memory by a set of partitioned matrices which are all capable of communicating to one-another downstream. In this paradigm, we'd like the user to be able to configure which state partitions are able to communicate with each other without having to write any new code.

For convenience, it seems sensible to also make the outputs from stochadex runs configurable. A user should be able to change the form of output that they want through, e.g., some specified function of $X$ at the time of outputting data. The times that the stochadex should output this data can also be decided by some user-specified condition so that the frequency of output is fully configurable as well. 

In summary, we've put together a schematic of configuration data types and their relationships in Fig.~\ref{fig:data-types-design}. In this diagram there is some indication of the data type that we propose to store each piece information in (in Go syntax), and the diagram as a whole should serve as a useful guide to the basic structure of configuration files for the stochadex.

It's clear that in order to simulate Eq.~(\ref{eq:x-step-def}), we need an interative algorithm which reapplies a user-specified function to the continually-updated history. But let's now return to the point we made earlier about how the performance of such an algorithm will depend on the size of the state history matrix $X$. The key bit of the algorithm design that isn't so straightforward is: how do we sucessfully split this state history up into separate partitions in memory while still enabling them to communicate effectively with each other? Other generalised simulation frameworks --- such as SimPy~\cite{simpy}, StoSpa~\cite{stospa} and FLAME GPU~\cite{flamegpu} --- have all approached this problem in different ways, and with different software architectures. 

In Fig.~\ref{fig:loop-design} we've illustrated what a loop involving separate state partitions looks like in the stochadex simulator. Each partition is handled by concurrently running execution threads of the same process, while a separate process may be used to handle the outputs from the algorithm. As the diagram shows, the main sequence of each loop iteration follows the pattern: 
%%
\begin{enumerate}
\item{The \texttt{PartitionCoordinator} requests more iterations from each state partition by sending an \texttt{IteratorInputMessage} to a concurrently running goroutine.}
\item{The \texttt{StateIterator} in each goroutine executes the iteration and stores the resulting state update in a variable.}
\item{Once all of the iterations have been completed, the \texttt{PartitionCoordinator} then requests each goroutine to update its relevant partition of the state history by sending another \texttt{IteratorInputMessage} to each.}
\end{enumerate}
%%
This pattern ensures that no partition has access to values in the state history which are out of sync with its current state in time, and hence prevents anachronisms from occuring in the overall state iteration. 

\begin{figure}[h]
\centering
\includegraphics[width=13cm]{images/chapter-1-stochadex-loop.drawio.png}
\caption{Schematic for a step of the stochadex simulation algorithm.}
\label{fig:loop-design}
\end{figure}

It's also worth noting that while Fig.~\ref{fig:loop-design} illustrates only a single process; it's obviously true that we may run many of these whole diagrams at once to parallelise generating independent realisations of the simulation, if necessary.

As we stated at the beginning of this chapter: the full implementation of the stochadex can be found on GitHub by following this link: \href{https://github.com/umbralcalc/stochadex}{https://github.com/umbralcalc/stochadex}. Users can build the main binary executable of this repository and determine what configuration of the stochadex they would like to have through config at runtime (one can infer these configurations from Fig.~\ref{fig:data-types-design}). As Go is a statically typed language, this level of flexibility has been achieved using code templating proceeding runtime build and execution via \texttt{go run} `under-the-hood'. Users who find this particular execution pattern undesirable can also use all of the stochadex types, tools and methods as part of a standard library import.

In order to debug the simulation code and gain a more intuitive understanding of the outputs from a model as it is being developed, we have also written a lightweight frontend dashboard React~\cite{react} app in TypeScript to visualise any stochadex simulation as it is running. This dashboard can be launched by passing config at runtime to the main stochadex executable, and we have illustrated how all this fits together in a flowchart shown in Fig.~\ref{fig:stochadex-main}.

\begin{figure}[h]
\centering
\includegraphics[width=9cm]{images/chapter-1-stochadex-main.drawio.png}
\caption{A diagram of the main stochadex binary executable.}
\label{fig:stochadex-main}
\end{figure}



\part*{
\includegraphics*[width=14cm]{images/page-design-2.png}}

\chapter{\sffamily Learning from ants on curved surfaces}

{\bfseries\sffamily Concept.} The idea here is 

\section{\sffamily Diffusive limits for ant interactions}

\part*{
\includegraphics*[width=14cm]{images/page-design-3.png}}

\chapter{\sffamily Quantum jumps on generic networks}

{\bfseries\sffamily Concept.} The idea is to follow this sort of thing \href{https://en.wikipedia.org/wiki/Quantum_jump_method}{here} to simulate the Lindblad equation over an arbitrary network of entangled states.

\section{\sffamily The Lindblad equation}

\part*{{\sffamily Part 2. {\color{gray75} How do we then learn/identify the answer to Part 1 from real-world data?}}
\includegraphics*[width=14cm]{images/page-design-1.png}}

\chapter{\sffamily Empirical dynamical emulators}

{\bfseries\sffamily Concept.} To extend the formalism that we developed in previous chapters to enable the empirical emulation of real-world data in a probabilistic way. This technique should enable a researcher to model complex dynamical trends in the data very well; at the cost of making the abstract interpretation of the model less immediately comprehensible than the statistical inference models in some proceeding chapters. As our generalised framework applies to a wide variety stochastic phenomena, our emulator will be applicable to a great breadth of data modeling problems as well. We will also explore some examples which illustrate how our empirical emulator should be applied in practice and then follow this up with how the code is designed and implemented as part of a new software package called `learnadex'. For the mathematically-inclined, this chapter will take a detailed look at how our formalism can be extended to focus on probabilistic dynamical process emulation. For the programmers, the software described in this chapter lives in the public Git repository: \href{https://github.com/umbralcalc/learnadex}{https://github.com/umbralcalc/learnadex}.

\section{\sffamily Probabilistic formalism}

The key distinction between the methods that we will develop in this chapter and the ones in the proceeding chapters is in their utility when faced with the problem of attempting to model real-world data. In the proceeding chapter, we shall describe some powerful techniques that can be used most effectively when the researcher is aware of the family of models that generated the data. In the present chapter, we will go into the details of how a more `empirical' approach can be derived for dynamical process modeling in a probabilistic framework which locally adapts the model to the data through time. 

While we think that it's worth going into some mathematical detail to give a better sense of where our formalism comes from; we want to emphasise that the framework we discuss here is not especially new to the technical literature. Our overall framework draws on influences from Empirical Dynamical Modeling (EDM)~\cite{sugihara1990nonlinear}, some classic nonparametric local regression techniques --- such as LOWESS/Savitzky-Golay filtering~\cite{savitzky1964smoothing} --- and perhaps Gaussian processes~\cite{murphy2012machine} as well. The novelties here, instead, lie more in the specifics of how we combine some of these ideas together when referencing the stochadex formalism, and how this manifests in designing more generally-applicable software for the user.

Before we are able to develop this empirical emulator, we need to return to the stochadex formalism that we introduced in the first chapter of this book. As we discussed at that point; this formalism is appropriate for sampling from nearly every stochastic phenomenon that one can think of. However, when trying robustly assess how far a model is from accurately describing a set of real-world data, trying to use only generated samples of the model process can be diffcult. Instead, in this section, we are going to extend this formalism to look at how probability theory can help with this data comparison problem in a systematic way.

So, how do we begin? In the first chapter, we defined the general stochastic process with the formula $X^{i}_{{\sf t}+1} = F^{i}_{{\sf t}+1}(X',z,{\sf t})$. This equation also has an implicit \emph{master equation} associated to it that fully describes the time evolution of the \emph{probability density function} $P_{{\sf t}+1}(x)$ of the most recent matrix row $x=X_{{\sf t}+1}$ at time ${\sf t}$. This can be written as
%%
\begin{align}
P_{{\sf t}+1}(x) &= \frac{1}{{\sf t}}\sum_{{\sf t}'=0}^{{\sf t}}\int_{\omega_{{\sf t}'}}{\rm d}x' P_{{\sf t}'}(x') P_{({\sf t}+1){\sf t}'}(x\vert x') \label{eq:master-x-cont} \,,
\end{align}
%%
where at the moment we are assuming the state space is continuous in each dimension and $P_{({\sf t}+1){\sf t}'}(x\vert x')$ is the conditional probability that the matrix row at time $({\sf t}+1)$ will be $x=X_{{\sf t}+1}$ given that the row at time ${\sf t}'$ was $x'=X_{{\sf t}'}$. This is a very general equation which should almost always apply to any continuous stochastic phenomenon we want to study in due course. To try and understand what this equation is saying we find it's helpful to think of an iterative relationship between probabilities; each of which is connected by their relative conditional probabilities. This kind of thinking is also illustrated in Fig.~\ref{fig:master-eqn}. Let's say we also wanted to program what this equation is saying as a function in Go. Using a Monte Carlo approximation for the integral domain, the code might look something like this.

\begin{lstlisting}[language=Go]
type StateVector  []float64

// returns a random draw of the possible state vectors at this timestep
func RandomPossibleStateVectors(timeStepNumber int) []StateVector {
    // return a slice of randomly-drawn possible state vectors 
    // corresponding to the integral domain at this timestep
}

// returns the conditional probability of the state vector at this timestep 
// given the value that the state vector had on a previous timestep
func StateVectorConditionalProbability(
    stateVector StateVector,
    timeStepNumber int,
    previousStateVector StateVector,
    previousTimeStepNumber int,
) float64 {
    // return the conditional probability value
}

// returns the probability of the state vector at this timestep
func StateVectorProbability(
    stateVector StateVector, 
    timeStepNumber int,
) float64 {
    prob := 0.0
    // loop over all the possible previous timesteps
    for t := 0; t < timeStepNumber; t++ {
        // loop over the randomly-drawn possible state vectors 
        // for this previous timestep
        possibleStateVectors := RandomPossibleStateVectors(t)
        for _, possibleStateVector := range possibleStateVectors {
            // note the recursion
            prob += StateVectorProbability(possibleStateVector, t)*
                StateVectorConditionalProbability(
                    stateVector,
                    timeStepNumber,
                    possibleStateVector, 
                    t,
                )
        }
        // normalisation for the Monte Carlo integration
        prob /= float64(len(possibleStateVectors))
    }
    // timestep normalisation
    prob /= float64(timeStepNumber)
    return prob
}
\end{lstlisting}

The factor of $1/{\sf t}$ in Eq.~(\ref{eq:master-x-cont}) is a normalisation factor --- this just normalises the sum of all probabilities to 1 given that there is a sum over ${\sf t}'$. Note that, if the process is defined over continuous time, we would need to replace 
%%
\begin{align}
\frac{1}{{\sf t}}\sum_{{\sf t}'=0}^{{\sf t}} \rightarrow \frac{1}{t({\sf t})}\sum_{{\sf t}'=0}^{{\sf t}}\delta t({\sf t}') \,.
\end{align}
%% 
But what is $\omega_{\sf t}$? You can think of this as just the domain of possible $x'$ inputs into the integral which will depend on the specific stochastic process we are looking at.

\begin{figure}[h]
\centering
\includegraphics[width=8cm]{images/master-eq-graph.drawio.png}
\caption{Graph representation of Eq.~(\ref{eq:master-x-cont}).}
\label{fig:master-eqn}
\end{figure} 

If we wanted to compute the mean of the distribution ${\rm E}_{{\sf t}+1}(x)$ in Eq.~(\ref{eq:master-x-cont}), it would be straightforward to just multiply both sides of the expression by $x$ and integrate over ${\rm d}x$ in the $\omega_{{\sf t}}$ domain. However, there is another similar expression for the mean that we can derive under certain conditions which will be valuable to us when developing the empirical emulator. If the probability distribution is \emph{stationary} --- meaning that $P_{{\sf t}'}(x)=P_{{\sf t}''}(x)$ for all ${\sf t}'$ and ${\sf t}''$ --- it's possible to derive\footnote{To see that this is true, first note that the joint distribution $P_{({\sf t}+1){\sf t}'}(x,x')=P_{({\sf t}+1){\sf t}'}(x\vert x')P_{{\sf t}'}(x')$. Secondly, note that joint distributions always allow variable swaps trivially like this $P_{({\sf t}+1){\sf t}'}(x,x')=P_{{\sf t}'({\sf t}+1)}(x',x)$. Then, lastly, note that stationarity of $P_{{\sf t}+1}(x)=P_{{\sf t}'}(x)$ means 
$${\rm E}_{{\sf t}+1}(x)=\int {\rm d}x\int {\rm d}x' \, x\,P_{({\sf t}+1){\sf t}'}(x,x')=\int {\rm d}x\int {\rm d}x'  \, x\, P_{{\sf t}'({\sf t}+1)}(x,x')=\int {\rm d}x\int {\rm d}x'  \, x\, P_{({\sf t}+1){\sf t}'}(x',x)\,,$$
where we've used the trivial variable swap to get to the last equality, and the domain references $\omega_{{\sf t}'}$ in the integrals are implicitly defined.} 
%%
\begin{align}
{\rm E}_{{\sf t}+1}(x) &= \frac{1}{{\sf t}}\sum_{{\sf t}'=0}^{{\sf t}}\int_{\omega_{{\sf t}'}}{\rm d}x'\int_{\omega_{{\sf t}+1}}{\rm d}x \, x' \, P_{{\sf t}'}(x') P_{({\sf t}+1){\sf t}'}(x\vert x') \,.
\end{align}
%%

In order to encode higher-order correlations between previous state vectors, we can extend Eq.~(\ref{eq:master-x-cont}) to consider a joint distribution in the past like this
%%
\begin{align}
P_{{\sf t}+1}(x) &= \frac{1}{{\sf t}}\sum_{{\sf t}'=0}^{{\sf t}}\frac{1}{{\sf t}'}\sum_{{\sf t}''=0}^{{\sf t}'}\int_{\omega_{{\sf t}'}}{\rm d}x'\int_{\omega_{{\sf t}''}}{\rm d}x'' P_{{\sf t}'{\sf t}''}(x', x'') P_{({\sf t}+1){\sf t}'{\sf t}''}(x\vert x', x'') \label{eq:master-x-cont-joint-pair} \,.
\end{align}
%%
This equation would apply if we wanted to retrieve how the out-of-time-order pairwise correlations between past values $x'$ at timestep ${\sf t}'$ and $x''$ at timestep ${\sf t}''$ affect the probability of the matrix row being $x$ at timestep ${\sf t}+1$.

What other processes can be described by Eq.~(\ref{eq:master-x-cont})? For Markovian phenomena, the equation no longer depends on timesteps older than the immediately previous one, hence the expression reduces to just
%%
\begin{align}
P_{{\sf t}+1}(x) &= \int_{\omega_{\sf t}}{\rm d}x' P_{\sf t}(x') P_{({\sf t}+1){\sf t}}(x\vert x') \label{eq:master-x-cont-markov} \,.
\end{align}
%%
An analog of Eq.~(\ref{eq:master-x-cont}) exists for discrete state spaces as well. We just need to replace the integral with a sum and the schematic would look something like this
%%
\begin{align}
P_{{\sf t}+1}(x) &= \frac{1}{{\sf t}}\sum_{{\sf t}'=0}^{\sf t}\sum_{\omega_{{\sf t}'}} P_{{\sf t}'}(x') P_{({\sf t}+1){\sf t}'}(x \vert x') \label{eq:master-x-disc} \,,
\end{align}
%%
where we note that the $P$'s in the expression above all now refer to \emph{probability mass functions}. In the even-simpler case where $x$ is just a vector of binary `on' or `off' states, Eq.~(\ref{eq:master-x-disc}) reduces to
%%
\begin{align}
P^i_{{\sf t}+1} &= \frac{1}{{\sf t}}\sum_{{\sf t}'=0}^{\sf t} \sum_{j=1}^d P^j_{{\sf t}'} P^{ij}_{({\sf t}+1){\sf t}'} = \frac{1}{{\sf t}}\sum_{{\sf t}'=0}^{\sf t} \sum_{j=1}^d \big[ P^j_{{\sf t}'} A^{ij}_{({\sf t}+1){\sf t}'} + (1-P^j_{{\sf t}'}) B^{ij}_{({\sf t}+1){\sf t}'} \big] \label{eq:master-x-disc-binary}\,,
\end{align}
%% 
where $P^i_{{\sf t}'}$ now represents the probability that element $x^i=1$ (is `on') at time ${\sf t}'$. The matrices $A$ and $B$ are defined as conditional probabilities where the previous state in time $P^j_{{\sf t}'}$ was either `on' or `off', respectively.

In this section, we looked into how the mathematical formalism used in the stochadex could be extended with probability theory. Now that we have more of a sense of how this formalism works, we are ready to move on to designing the algorithms for our emulator. So let's go!

\section{\sffamily Emulator algorithms}

Helpful to write the basic structure of algorithm out in Go.

\section{\sffamily Software design}



\part*{
\includegraphics*[width=14cm]{images/page-design-1.png}}

\chapter{\sffamily Inferring dynamical 2D maps}

{\bfseries\sffamily Concept.} The idea here is... For the mathematically-inclined, this chapter will... For the programmers, the software described in this chapter lives in the public Git repository: \href{https://github.com/umbralcalc/learnadex}{https://github.com/umbralcalc/learnadex}.


\section{\sffamily Adapting the probabilistic formalism}

The extra detail that's needed here is to consider how we encode a 2d spatial process into our state vector, and how the elements of the resulting state vector might be correlated to one another depending on their spatial proximity. If we start with a Markovian Gaussian random field, we can derive the Mat\'{e}rn kernel over these spatial coordinates in order to correlate the state vectors in such a way.

\part*{
\includegraphics*[width=14cm]{images/page-design-1.png}}

\chapter{\sffamily Learning from ants on curved surfaces}

{\bfseries\sffamily Concept.} The idea here is 

\section{\sffamily Diffusive limits for ant interactions}

\part*{
\includegraphics*[width=14cm]{images/page-design-1.png}}

\chapter{\sffamily A world of hydrodynamic ensembles}

{\bfseries\sffamily Concept.} The idea here is 


\section{\sffamily The Boltzmann/Navier-Stokes equations}

\part*{
\includegraphics*[width=14cm]{images/page-design-1.png}}

\chapter{\sffamily Generalised statistical inference tools}

{\bfseries\sffamily Concept.} The idea here is to extend the stochadex with tools for very generalised statistical inference of an explicit model (Bayesian and Frequentist - both with likelihood likelihood-free ABC algorithms and the like) that will work in nearly every situation where the user knows the model space. Probably need to exploit the phase space analogy of the formalism.

\section{\sffamily Likelihood-free methods}

\part*{{\sffamily Part 3. {\color{gray75} How do we simulate a general set of control policies to interact with the answer to Part 1?}}
\includegraphics*[width=14cm]{images/page-design-1.png}}

\chapter{\sffamily Interacting with systems in general}

{\bfseries\sffamily Concept.} To design and build a software which can interact with stochastic processes of any kind, either manually through user input or automatically by introducing a `policy'. The mathematical formalism and software that we introduce here will serve as a common language and interface for any simulation studies into manipulating real world phenomena, and will enable the learning of control algorithms in later chapters of this book. We will implement this new interaction software as an extension to the stochadex package. For the mathematically-inclined, this chapter will cover how interactions are structured in theory by developing some useful extensions to the stochadex formalism and illustrating with some simple examples. For the programmers, the public Git repository for the code described in this chapter can be found here: \href{https://github.com/umbralcalc/stochadex}{https://github.com/umbralcalc/stochadex}.

\section{\sffamily Formalising general interactions}

Let's start by considering how we might adapt the mathematical formalism we have been using so far to be able to take actions which can manipulate the state at each timestep. Using the mathematical notation that we inherited from the stochadex, we may extend the formula for updating the state history matrix $X'\rightarrow X$ to include a new layer of possible interactions which is facilitated by a new vector-valued `state action' function $G_{{\sf t}}$. In doing so we shall be defining the domain of an acting entity in the stochastic process environment --- which we shall hereafter refer to as simply the `agent'.

During a timestep over which actions are performed by the agent, the stochadex state update formula can be extended to look to include interactions by composition with the original state update function like so
%%
\begin{align}
X_{{\sf t}+1}^i &= G^i_{{\sf t}+1}[F_{{\sf t}+1}(X', z, {\sf t}), {\cal Z}_{{\sf t}+1}, {\cal A}_{{\sf t}+1}] = {\cal F}^i_{{\sf t}+1}(X', z, {\cal Z}_{{\sf t}+1}, {\cal A}_{{\sf t}+1}, {\sf t}) \label{eq:generalised-state-actions} \,,
\end{align}
%%
where we have also introduced the concept of the `actions' performed ${\cal A}_{{\sf t}+1}$ on the system; some vector of parameters which define what actions are taken at timestep ${\sf t}+1$. In this equation notice that, in addition to the constant vector of parameters $z$ (as in the stochadex formalism), we have introduced a new time-dependent vector of other parameters $Z_{{\sf t}+1}$ that may be updated by the agent at any (but not necessarily every) timestep.

The code for the new iteration formula given by Eq.~(\ref{eq:generalised-state-actions}), which includes taking actions in the same timestep, would look something like Fig.~\ref{fig:iterations-with-actions}.

\begin{figure}[h]
\centering
\includegraphics[width=11cm]{images/chapter-9-iterations-with-actions.drawio.png}
\caption{Code schematic of Eq.~(\ref{eq:generalised-state-actions}).}
\label{fig:iterations-with-actions}
\end{figure}

\textcolor{red}{Rewrite this section to remove the existence of parametric actions, rewrite the code in the stochadex to match, and make this section more about the kinds of actions that can be performed on the basic stochadex simulations.}

\textcolor{red}{Include a full description of policies here!!}

Up to this point, we have only considered actions which were either scheduled up front through some fixed process or through user interaction via a game interface. In order to start creating algorithms to act on the system state for us, we now need to develop a formalism which `closes the loop' by feeding information back from the stochastic process to another decision-making process. Note that in most cases, the state of real-world phenomena cannot be measured perfectly. So in order to enable any agent trained on simulated phenomena to potentially act in the real world, we will need to model this measurement process as part of the information retrieval step.

Let's now define the concept of an `environment state' ${\cal S}_{{\sf t}+1}$ at timestep ${\sf t}+1$; this is a new vector that doesn't have to share the same length as the measured state vector $X_{{\sf t}+1}$. We will then say generally that this environment state is `observed' by the agent using the following observation function
%%
\begin{align}
{\cal S}_{{\sf t}+1}^i &= O_{{\sf t}+1}^i(X',{\cal Z}_{{\sf t}+1},{\sf t}) \label{eq:generalised-state-measurement} \,,
\end{align}
%%
where we have also introduced a new vector ${\cal Z}_{{\sf t}+1}$ which we will use to store all of the relevant parameters to the agent\footnote{This vector is intended to include parameters for measurement, policy specification and ultimately the learning algorithm as well.} at timestep ${\sf t}+1$.

If we are now given the conditional probability that an action vector element ${\cal A}_{{\sf t}+1}=a$ is chosen given that state vector ${\cal S}_{{\sf t}+1}=s$ has been measured $\pi (a,s) = p(a\vert s)$, we can use this to draw new actions for the agent with a newly defined action-generating function
%%
\begin{align}
{\cal A}_{{\sf t}+1}^i &= \Pi_{{\sf t}+1}^i({\cal S}_{{\sf t}+1}, {\cal Z}_{{\sf t}+1}) \label{eq:action-generating-function} \,.
\end{align}
%%
From this point on we'll call $\pi (a,s)$ the `policy' adoped by the agent. A Markov Decision Process (MDP) defines an algorithm in which the agent uses a single state measurement vector and its given policy $\pi$ to draw actions ${\cal A}_{{\sf t}+1}$ at timestep ${{\sf t}+1}$. It then performs these actions in its environment, which we have previously formalised through defining the iteration $X_{{\sf t}+1} = {\cal F}_{{\sf t}+1}(X',Z_{\sf t},{\cal A}_{{\sf t}+1},{\sf t})$. 

\part*{
\includegraphics*[width=14cm]{images/page-design-1.png}}

\chapter{\sffamily Learning from ants on curved surfaces}

{\bfseries\sffamily Concept.} The idea here is 

\section{\sffamily Diffusive limits for ant interactions}

\part*{
\includegraphics*[width=14cm]{images/page-design-1.png}}

\chapter{\sffamily Managing a Rugby match}

{\bfseries\sffamily Concept.} The idea here is 

\section{\sffamily Introduction}

Since the basic game engine will run using the \href{https://github.com/umbralcalc/stochadex}{stochadex} sampler, the novelties in this project are all in the design of the rugby match model itself. And, in this instance, we're not especially keen on spending a lot of time doing detailed data analysis to come up with the most realistic values for the parameters that are dreamed up here. Even though this would also be interesting.

One could do this data analysis, for instance, by scraping player-level performance data from one of the excellent websites that collect live commentary data such as \href{https://www.rugbypass.com/}{rugbypass.com} or \href{https://www.espn.co.uk/rugby/}{espn.co.uk/rugby}.

This game is primarily a way of testing out the interface of the stochadex for other users to build projects with. This should help to both iron out some of the kinks in the design, as well as prioritise adding some more convenience methods for event-based modelling into its code base.

\section{\sffamily Designing the event simulation engine}

We need to begin by specifying an appropriate event space to live in when simulating a rugby match. It is important at this level that events are defined in quite broadly applicable terms, as it will define the state space available to our stochastic sampler and hence the simulated game will never be allowed to exist outside of it. So, in order to capture the fully detailed range of events that are possible in a real-world match, we will need to be a little imaginative in how we define certain gameplay elements when we move through the space.

The diagrams below sum up what should hopefully work as a decent initial approximation while providing a little context with specific examples of play action.

\begin{figure}[h]
\includegraphics[width=8cm]{images/test.drawio.png}
\caption{Simplified event graph of a rugby union match - replace with drawio.}
\label{fig:event-graph}
\end{figure}

\begin{figure}[h]
\includegraphics[width=10cm]{images/test.drawio.png}
\caption{Optional model ideas - replace with drawio.}
\label{fig:model-ideas}
\end{figure}

\section{\sffamily Linking to player attributes}

\section{\sffamily Deciding on gameplay actions}

\section{\sffamily Writing the game itself}

\part*{
\includegraphics*[width=14cm]{images/page-design-1.png}}

\chapter{\sffamily Influencing house prices}

{\bfseries\sffamily Concept.} The idea here is 

\part*{
\includegraphics*[width=14cm]{images/page-design-1.png}}

\part*{{\sffamily  Part 4. {\color{gray75} How do we then optimise the answer to Part 3 to achieve a specified control objective?}}
\includegraphics*[width=14cm]{images/page-design-1.png}}

\chapter{\sffamily Optimising actions for control objectives}

{\bfseries\sffamily Concept.} The idea here is 

\section{\sffamily States, actions and attributing rewards}

\textcolor{red}{Rewrite the beginning of this section to talk about:
\begin{itemize}
\item{remove the need to parametric actions from this section --- actions are encoded as their own separate state variables in the history and the policy can be defined through $z$ which can change through a data update}
\item{using the online learning of state and parameters MAP}
\item{forward-predicting with this ensemble given an ensemble of policies to try, applying a discount to the forecasts to compute cumulative reward and then choosing the best route}
\item{once this pattern is working, the methodology can be adapted to improve the policy/action-generating function}    
\end{itemize}
}

In order to assess the quality of an agent's actions, we might later attribute a reward value ${\cal R}_{{\sf t}}$ for actions that were taken at timestep ${\sf t}$. Using a series of these rewards, a return value $R$ can also be computed using a future discount factor $\gamma$ like so 
%%
\begin{align}
R &= \sum_{{\sf t}=0}^\infty \gamma^{\sf t}{\cal R}_{\sf t} \,.
\end{align}
%%
A state-value function $V_\pi$ is defined as the expectation (under policy $\pi$) of return $R$, given state vector ${\cal S}_{\sf t}=s$, i.e.,
%%
\begin{align}
V_\pi (s) = {\rm E}_\pi (R\vert s) \,.
\end{align}
%%
Similarly, an action-value function $Q_\pi$ is defined as the expectation (again, under policy $\pi$) of return $R$, given state vector ${\cal S}_{\sf t}=s$ and action vector ${\cal A}_{\sf t}=a$, i.e.,
%%
\begin{align}
Q_\pi (s,a) = {\rm E}_\pi (R\vert s,a) \,.
\end{align}
%%

Follow-up this bit with the model-based approach that we're going to take in this book.
\begin{itemize}
\item{Talk through value and policy learning - in this book we will be doing the value learning with our generalised stochastic model and then the policy learning bit is more nuanced.}
\item{The value learning can be facilitated in software using a predictive model which is able to roll forecast rewards forward in time in a Monte Carlo fashion up to a window from a certain point given an input prior distribution of policies.}
\item{This input prior distribution of policies can itself be optimised by maximising expected utility in a Bayesian design framework!}
\end{itemize}

\part*{
\includegraphics*[width=14cm]{images/page-design-1.png}}

\chapter{\sffamily Resource allocation for epidemics}

{\bfseries\sffamily Concept.} The idea here is to limit the spread of some abstract epidemic through the correct time-dependent resource allocation.

\part*{
\includegraphics*[width=14cm]{images/page-design-1.png}}

\chapter{\sffamily Quantum system control}

{\bfseries\sffamily Concept.} The idea here is to follow stuff along these lines \href{https://arxiv.org/pdf/1210.7127.pdf}{here}. And also this sort of thing \href{https://en.wikipedia.org/wiki/Quantum_jump_method}{here} to simulate the Lindblad equation over an arbitrary network of entangled states.
 


%\appendix
%\chapter{First and only appendix}
\backmatter
\bibliographystyle{JHEP}
\bibliography{book}
\end{document}