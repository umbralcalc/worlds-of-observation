\documentclass{book}

\usepackage{etoolbox}

\makeatletter
\def\subtitle#1{\gdef\@subtitle{#1}}
\patchcmd\maketitle
  {{\LARGE \@title \par}}
  {{\LARGE \@title \par}%
   \vskip 1.5em
   {\Large \@subtitle \par}}
\makeatother

\usepackage[utf8x]{inputenc}
\usepackage{amsmath,mathtools,bbm}
\usepackage{amsfonts}
\usepackage{amssymb}
\usepackage{graphicx}
\usepackage{listings}
\usepackage{appendix}
\usepackage{enumitem}
\usepackage{bm}
\usepackage{multicol}
\usepackage{geometry}
\usepackage{colortbl}
\usepackage{changepage}
\usepackage{color}
\usepackage{mathrsfs}
\usepackage{bigints}
\usepackage{pdflscape}
\usepackage{adjustbox}
\usepackage{tocloft}
\usepackage{lscape}
\usepackage[colorlinks=true,
            linkcolor=blue,
            urlcolor=blue,
            citecolor=blue]{hyperref}

\lstset{%
backgroundcolor=\color[gray]{.85},
basicstyle=\small\ttfamily,
breaklines = true,
keywordstyle=\color{red!75},
columns=fullflexible,
}%

\lstdefinelanguage{BibTeX}
  {keywords={%
      @article,@book,@collectedbook,@conference,@electronic,@ieeetranbstctl,%
      @inbook,@incollectedbook,@incollection,@injournal,@inproceedings,%
      @manual,@mastersthesis,@misc,@patent,@periodical,@phdthesis,@preamble,%
      @proceedings,@standard,@string,@techreport,@unpublished%
      },
  comment=[l][\itshape]{@comment},
  sensitive=false,
  }

\definecolor{codegreen}{rgb}{0,0.6,0}
\definecolor{codegray}{rgb}{0.5,0.5,0.5}
\definecolor{codepurple}{rgb}{0.58,0,0.82}
\definecolor{backcolour}{rgb}{0.97,0.97,0.97}
\definecolor{stringorange}{rgb}{0.95, 0.4, 0.18}

\lstdefinestyle{mystyle}{
    backgroundcolor=\color{backcolour},   
    commentstyle=\color{codegreen},
    keywordstyle=\color{codepurple},
    numberstyle=\tiny\color{codegray},
    stringstyle=\color{stringorange},
    basicstyle=\ttfamily\footnotesize,
    breakatwhitespace=false,         
    breaklines=true,                 
    captionpos=b,                    
    keepspaces=true,                 
    numbers=left,                    
    numbersep=5pt,                  
    showspaces=false,                
    showstringspaces=false,
    showtabs=false,                  
    tabsize=2
}
\lstset{style=mystyle}

\sloppy
\usepackage{tikz, lipsum}% http://ctan.org/pkg/{pgf,lipsum}
\newcommand*{\chapnumfont}{\normalfont\sffamily\huge\bfseries}
\newcommand*{\printchapternum}{
  \begin{tikzpicture}
    \draw[fill,color=gray75] (0,0) rectangle (2cm,2cm);
    \draw[color=white] (1cm,1cm) node { \chapnumfont\thechapter };
  \end{tikzpicture}
}
\newcommand*{\chaptitlefont}{\normalfont\sffamily\Huge\bfseries}
\newcommand*{\printchaptertitle}[1]{\flushright\chaptitlefont#1}

\makeatletter
% \@makechapterhead prints regular chapter heading.
% Taken directly from report.cls and modified.
\def\@makechapterhead#1{%
  \vspace*{50\p@}%
  {\parindent \z@ \raggedleft
    \ifnum \c@secnumdepth >\m@ne
        \printchapternum
        \par\nobreak
        \vskip 20\p@
    \fi
    \interlinepenalty\@M
    \printchaptertitle{#1}\par\nobreak
    \vskip 40\p@
  }}
% \@makeschapterhead prints starred chapter heading.
% Taken directly from report.cls and modified.
\def\@makeschapterhead#1{%
  \vspace*{50\p@}%
  {\parindent \z@ \raggedleft
    \interlinepenalty\@M
    \printchaptertitle{#1}\par\nobreak
    \vskip 40\p@
  }}
\makeatother

\makeatletter\@addtoreset{chapter}{part}\makeatother%

\renewcommand{\cfttoctitlefont}{\hfill\Huge\bfseries\sffamily}

\renewcommand\contentsname{Table of contents}

\definecolor{gray75}{gray}{0.75}

\title{\Huge \bfseries\sffamily Worlds Of Observation}
\subtitle{\Large \bfseries\sffamily \color{gray75} Building more realistic environments for machine learning}
\author{\bfseries\sffamily Robert J. Hardwick}
\date{\today}

\begin{document}
\begin{titlepage}
\centering
\vspace*{1.5\baselineskip}
{\color{gray75}\rule{13cm}{1.3pt}}\vspace*{-\baselineskip}\vspace*{2pt} % Thick horizontal rule
{\color{gray75}\rule{13cm}{0.4pt}} \\ % Thin horizontal rule
\vspace{1.2\baselineskip} % Whitespace 
{\Huge \bfseries\sffamily Worlds Of Observation} \\ 
\vspace{4mm}
{\Large \bfseries\sffamily \color{gray75} Building more realistic environments\\ for machine learning \\}
\vspace*{0.75\baselineskip}
{\color{gray75}\rule{13cm}{0.4pt}}\vspace*{-\baselineskip}\vspace*{2.75pt} % Thick horizontal rule
{\color{gray75}\rule{13cm}{1.3pt}} \\ % Thin horizontal rule
\vspace{1.0\baselineskip} % Whitespace 
{\large \bfseries\sffamily Robert J. Hardwick \\
\vspace*{1.2\baselineskip}}
\today
\vfill
Edited by C. M. Gomez-Perales \\ \vspace{1mm}
Shared by the author under an \href{https://opensource.org/licenses/MIT}{MIT License}
\end{titlepage}

\chapter*{Introduction}

In many systems of practical importance to the real world it's common to find that observing the state of the system itself is only partially possible. One needs to only think of the measurement uncertainties in any scientific experiment, the latent demand behind orders in a financial market, the unknown reservoirs of infection for a disease pathogen or even the limits to complete supply chain component observability in recognising just how ubiquitous this situation is. This obscurity can make the learning of algorithms to control these systems using observations alone an extreme --- if not sometimes impossible --- challenge without further insight provided by a model.

\emph{Worlds Of Observation} is a book about building more realistic training environments for machine learning algorithms to control these `noisy' systems in the real world. While model-free reinforcement learning is a popular and very powerful approach to generating such algorithms~\cite{sutton2018reinforcement} (especially when there is plenty of data and the system is fully observable), this book will primarily explore how to create control algorithms using a more model-based approach. Those readers who are data scientists, research engineers, statistical programmers or computational scientists may find our mathematically descriptive, yet practically-minded, approach in this book quite interesting and maybe a little different to the usual perspectives.

As can be expected from a book about algorithms, this text accompanies a lot of new open-source scientific software written in various combinations of Go~\cite{golang}, C++~\cite{c++lang}, Python~\cite{pythonlang} and TypeScript~\cite{typescriptlang}. A major motivation for creating these new tools is to prepare a foundation of code from which to develop new and more complex applications. We also hope that the resulting framework will enable anyone to study new phenomena and explore complex control problems effectively, regardless of their scientific background.

The need to properly test all this software has also provided a wonderful excuse to study and play with an extensive range of models which simulate real-world systems. We've chosen these models based on a fairly broad background of interests, but also to illustrate the cross-disciplinary applicability of our algorithmic framework. To achieve this generalisation, an essential part of this book is the mathematical framework that it introduces and uses throughout. To explore these simulated systems further we encourage readers to take a look at WorldsOOp (\href{https://worldsoop.github.io/}{https://worldsoop.github.io/}), which is an open source software ecosystem inspired directly by this book.

It seems silly to us that mathematical formalities can obscure the practical computations that a programmer is being asked to implement when reading an equation. So, while we've tried to be as ambitious as possible with the level of technical detail in this book, we've also attempted to write many of the mathematical expressions in a more computer-friendly way where feasible,\footnote{For example, we'll typically be thinking more in terms of `matrices' and less about `operators'.} in contrast with the more conventional formal descriptions. To help with this goal of explainability, we also make use of quite a lot of illustrations and diagrams.

A quick note on the code: any software that we describe in this book (including the software which compiles the book itself~\cite{worldsofobservationbookgithub}) will always be shared under a MIT License~\cite{mitlicense} in a public Git repository.\footnote{The repositories will always be somewhere on these lists: \href{https://github.com/umbralcalc?tab=repositories}{https://github.com/umbralcalc?tab=repositories}, \href{https://github.com/orgs/worldsoop/repositories}{https://github.com/orgs/worldsoop/repositories}.} Forking these repositories and submitting pull requests for new features or applications is strongly encouraged too, though we apologise in advance if we don't follow these up very quickly as all of this work has to be conducted independently in free time, outside of work hours.

The book is split into two main parts: {\bfseries\sffamily Part 1}, which details the theoretical background and design of code; and {\bfseries\sffamily Part 2}, which explores all of the applications to realistic examples that we wanted to initially try. We hope you, the reader, really enjoy reading through this book and using the all of the code that was built while writing it. We're very grateful to have been able to make use of all the amazing open source software which would otherwise have made this project impossible to achieve.

To cite this book in any work, please use the following BibTeX:
%%
\begin{adjustwidth}{5em}{5em}
\begin{lstlisting}[language=BibTeX,numbers=none]
@article{worlds-of-observation-2023,
  title   = {Worlds Of Observation: Building more realistic environments for machine learning},
  author  = {Hardwick, Robert J.},
  journal = {umbralcalc.github.io},
  year    = {2023},
  url     = {https://umbralcalc.github.io/worlds-of-observation/book.pdf},
}
\end{lstlisting}
\end{adjustwidth}
%%

\newpage \ \newpage
{\sffamily \tableofcontents}
\mainmatter

\part*{{\sffamily Part 1}}

\chapter{\sffamily Building a generalised simulator}

{\bfseries\sffamily Concept.} To design and build a generalised simulation engine that is able to generate samples from practically any real-world stochastic processes that a researcher could encounter. With such a thing pre-built and self-contained, it can become the basis upon which to build generalised software solutions for a lot of different problems. For the mathematically-inclined, this chapter will require the introduction of a new formalism which we shall refer back to throughout the book. For the programmers, the public Git repository for the code that is described in this chapter can be found here: \href{https://github.com/umbralcalc/stochadex}{https://github.com/umbralcalc/stochadex}.

\section{\sffamily Computational formalism}

Before we dive into software design we need to mathematically define the general computational approach that we're going to take. Because the language of stochastic processes is primarily mathematics, we'd argue this step is essential in enabling a really general description. From experience, it seems reasonable to start by writing down the following formula which describes iterating some arbitrary process forward in time (by one finite step) and adding a new row each to some matrix $X_{0:{\sf t}} \rightarrow X_{0:{\sf t}+1}$
%%
\begin{align}
X^{i}_{{\sf t}+1} &= F^{i}_{{\sf t}+1}(X_{0:{\sf t}},z,{\sf t}) \,, \label{eq:x-step-def}
\end{align}
%%
where: $i$ is an index for the dimensions of the `state' space; ${\sf t}$ is the current time index for either a discrete-time process or some discrete approximation to a continuous-time process; $X_{0:{\sf t}+1}$ is the next version of $X_{0:{\sf t}}$ after one timestep (and hence one new row has been added); $z$ is a vector of arbitrary size which contains the `hidden' other parameters that are necessary to iterate the process; and $F^i_{{\sf t}+1}(X_{0:{\sf t}},z,{\sf t})$ as the latest element of an arbitrary matrix-valued function. 

Throughout the book, the notation $A_{{{\sf b}:{\sf c}}}$ will always refer to a slice of rows from index ${\sf b}$ to ${\sf c}$ in a matrix (or row vector) $A$. As we shall discuss shortly, $F^i_{{\sf t}+1}(X_{0:{\sf t}},z,{\sf t})$ may represent not just operations on deterministic variables, but also on stochastic ones. There is also no requirement for the function to be continuous.

\begin{figure}[h]
\centering
\includegraphics[width=10cm]{images/chapter-1-fundamental-loop.drawio.png}
\caption{Graph representation of Eq.~(\ref{eq:x-step-def}).}
\label{fig:fundamental-loop}
\end{figure}

The basic computational idea here is illustrated in Fig.~\ref{fig:fundamental-loop}; we iterate the matrix $X$ forward in time by a row, and use its previous version $X_{0:{\sf t}}$ as an entire matrix input into a function which populates the elements of its latest rows. In pseudocode you could easily write something with the same idea in it, and it would probably look something like the method diagram in Fig.~\ref{fig:fundamental-loop-code}.

\begin{figure}[h]
\centering
\includegraphics[width=10cm]{images/chapter-1-fundamental-loop-code.drawio.png}
\caption{Pseudocode representation of Eq.~(\ref{eq:x-step-def}).}
\label{fig:fundamental-loop-code}
\end{figure}

Pretty simple! But why go to all this trouble of storing matrix inputs for previous values of the same process? It's true that this is mostly redundant for \emph{Markovian} phenomena, i.e., processes where their only memory of their history is the most recent value they took. However, for a large class of stochastic processes a full memory\footnote{Or memory at least within some window.} of past values is essential to consistently construct the sample paths moving forward. This is true in particular for \emph{non-Markovian} phenomena, where the latest values don't just depend on the immediately previous ones but can depend on values which occured much earlier in the process as well.

For more complex physical models and integrators, the distinct notions of `numerical timestep' and `total elapsed continuous time' will crop up quite frequently. Hence, before moving on further details, it will be important to define the total elapsed time variable $t({\sf t})$ for processes which are defined in continuous time. Assuming that we have already defined some function $\delta t({\sf t})$ which returns the specific change in continuous time that corresponds to the step ${\sf t}-1 \rightarrow {\sf t}$, we will always be able to compute the total elapsed time through the relation
%%
\begin{align}
t({\sf t}) &= \sum^{{\sf t}}_{{\sf t}'=0}\delta t({\sf t}') \label{eq:t-steps-sum} \,.
\end{align}
%%
This seems a lot of effort, no? Well it's important to remember that our steps in continuous time may not be constant, so by defining the $\delta t({\sf t})$ function and summing over it we can enable this flexibility in the computation. In case the summation notation is no fun for programmers; we're simply adding up all of the differences in time to get a total. We've illustrated this in Fig.~\ref{fig:time-step-summation} for more clarity.

\begin{figure}[h]
\centering
\includegraphics[width=10cm]{images/chapter-1-time-step-summation.drawio.png}
\caption{Illustration of Eq.~(\ref{eq:t-steps-sum}).}
\label{fig:time-step-summation}
\end{figure}

So, now that we've mathematically defined a really general notion of iterating the stochastic process forward in time, it makes sense to discuss some simple examples. For instance, it is frequently possible to split $F$ up into deteministic (denoted $D$) and stochastic (denoted $S$) matrix-valued functions like so
%%
\begin{align}
& F^{i}_{{\sf t}+1}(X_{0:{\sf t}},z,{\sf t}) = D^{i}_{{\sf t}+1}(X_{0:{\sf t}},z,{\sf t}) + S^{i}_{{\sf t}+1}(X_{0:{\sf t}},z,{\sf t}) \,.
\end{align}
%%
In the case of stochastic processes with continuous sample paths, it's also nearly always the case with mathematical models of real-world systems that the deterministic part will at least contain the term $D^{i}_{{\sf t}+1}(X_{0:{\sf t}},z,{\sf t}) = X^i_{\sf t}$ because the overall system is described by some stochastic differential equation. This is not a really requirement in our general formalism, however.

What about the stochastic term? For example, if we wanted to consider a \emph{Wiener process noise}, we can define $W^i_{{\sf t}}$ is a sample from a Wiener process for each of the state dimensions indexed by $i$ and our formalism becomes
%%
\begin{align}
& S^{i}_{{\sf t}+1}(X_{0:{\sf t}},z,{\sf t}) = W^i_{{\sf t}+1}-W^i_{\sf t} \label{eq:wiener}\,.
\end{align}
%%
One draws the increments $W^i_{{\sf t}+1}-W^i_{\sf t}$ from a normal distribution with a mean of $0$ and a variance equal to the length of continuous time that the step corresponded to $\delta t({\sf t}+1)$, i.e., the probability density $P_{{\sf t}+1}(x^i)$ of the increments $x^i=W^i_{{\sf t}+1}-W^i_{\sf t}$ is
%%
\begin{align}
P_{{\sf t}+1}(x^i) &= {\sf NormalPDF}[x^i;0,\delta t({\sf t}+1)] \,.
\end{align}
%%
Note that for state spaces with dimensions $>1$, we could also allow for non-trivial cross-correlations between the noises in each dimension. In pseudocode, the Wiener process is schematically represented by Fig.~\ref{fig:wiener-process}.

\begin{figure}[h]
\centering
\includegraphics[width=8cm]{images/chapter-1-wiener-process.drawio.png}
\caption{Schematic of code for a Wiener process.}
\label{fig:wiener-process}
\end{figure}

In another example, to model \emph{geometric Brownian motion noise} we would simply have to multiply $X^i_{\sf t}$ to the Wiener process like so
%%
\begin{align}
& S^{i}_{{\sf t}+1}(X_{0:{\sf t}},z,{\sf t}) = X^i_{\sf t}(W^i_{{\sf t}+1}-W^i_{\sf t})\label{eq:gbm} \,.
\end{align}
%%
Here we have implicitly adopted the Itô interpretation to describe this stochastic integration. Given a carefully-defined integration scheme other interpretations of the noise would also be possible with our formalism too, e.g., Stratonovich\footnote{Which would implictly give $S^{i}_{{\sf t}+1}(X_{0:{\sf t}},z,{\sf t}) = (X^i_{{\sf t}+1}+X^i_{\sf t})(W^i_{{\sf t}+1}-W^i_{\sf t}) / 2$ for Eq.~(\ref{eq:gbm}).} or others within the more general `$\alpha$-family'~\cite{van1992stochastic,risken1996fokker,rog-will-2000}. The pseudocode for any of these should hoepfully be fairly straightforward to deduce based on the lines we've already written above.

\begin{figure}[h]
\centering
\includegraphics[width=12cm]{images/chapter-1-ito-lemma.drawio.png}
\caption{Schematic of code for Eq.~(\ref{eq:general-wiener}).}
\label{fig:ito-lemma}
\end{figure}

We can imagine even more general processes that are still Markovian. One example of these in a single-dimension state space would be to define the noise through some general function of the Wiener process like so
%%
\begin{align}
S^0_{{\sf t}+1}(X_{0:{\sf t}},z,{\sf t}) &= g[W^0_{{\sf t}+1},t({\sf t}+1)]-g[W^0_{\sf t}, t({\sf t})] \\
&= \bigg[ \frac{\partial g}{\partial t} + \frac{1}{2}\frac{\partial^2 g}{\partial x^2} \bigg] \delta t ({\sf t}+1) + \frac{\partial g}{\partial x} (W^0_{{\sf t}+1}-W^0_{\sf t}) \label{eq:general-wiener}\,,
\end{align}
%%
where $g(x,t)$ is some continuous function of its arguments which has been expanded out with Itô's Lemma on the second line. Note also that the computations in Eq.~(\ref{eq:general-wiener}) could be performed with numerical derivatives in principle, even if the function were extremely complicated. This is unlikely to be the best way to describe the process of interest, however, the mathematical expressions above can still be made a bit more meaningful to the programmer in this way. The pseudocode in general would look something like Fig.~\ref{fig:ito-lemma}.

Let's now look at a more complicated type of noise. For example, we might consider sampling from a \emph{fractional Brownian motion} process $[B_{H}]_{\sf t}$, where $H$ is known as the `Hurst exponent'. Following Ref.~\cite{decreusefond1999stochastic}, we can simulate this process in one of our state space dimensions by modifying the standard Wiener process by a fairly complicated integral factor which looks like this
%%
\begin{align}
S^{0}_{{\sf t}+1}(X_{0:{\sf t}},z,{\sf t}) &= \frac{(W^0_{{\sf t}+1} - W^0_{\sf t})}{\delta t({\sf t})}\int^{t({\sf t}+1)}_{t({\sf t})}{\rm d}t' \frac{(t'-t)^{H-\frac{1}{2}}}{\Gamma (H+\frac{1}{2})} {}_2F_1 \bigg( H-\frac{1}{2};\frac{1}{2}-H;H+\frac{1}{2};1-\frac{t'}{t}\bigg) \label{eq:fbm} \,,
\end{align}
%%
where $S^{0}_{{\sf t}+1}(X_{0:{\sf t}},z,{\sf t})=[B_{H}]_{{\sf t}+1}-[B_{H}]_{{\sf t}}$. The integral in Eq.~(\ref{eq:fbm}) can be approximated using an appropriate numerical procedure (like the trapezium rule, for instance). In the expression above, we have used the symbols ${}_2F_1$ and $\Gamma$ to denote the ordinary hypergeometric and gamma functions, respectively. A computational form of this integral is illustrated in Fig.~\ref{fig:fractional-brownian-motion} to try and disentangle some of the mathematics as a program.

\begin{figure}[h]
\centering
\includegraphics[width=11cm]{images/chapter-1-fractional-brownian-motion.drawio.png}
\caption{Schematic of code for Eq.~(\ref{eq:fbm}).}
\label{fig:fractional-brownian-motion}
\end{figure}

So far we have mostly been discussing noises with continuous sample paths, but we can easily adapt our computation to discontinuous sample paths as well. For instance, \emph{Poisson process noises} would generally take the form
%%
\begin{align}
S^{i}_{{\sf t}+1}(X_{0:{\sf t}},z,{\sf t}) &= [N_{\lambda}]^i_{{\sf t}+1}-[N_{\lambda}]^i_{\sf t}\,,
\end{align}
%%
where $[N_{\lambda}]^i_{\sf t}$ is a sample from a Poisson process with rate $\lambda$. One can think of this process as counting the number of events which have occured up to the given interval of time, where the intervals between each succesive event are exponentially distributed with mean $1/\lambda$. Such a simple counting process could be simulated exactly by explicitly setting a newly-drawn exponential variate to the next continuous time jump ${\delta t}({\sf t}+1)$ and iterating the counter. Other exact methods exist to handle more complicated processes involving more than one type of `event', such as the Gillespie algorithm~\cite{gillespie1977exact} --- though these techniques are not always be applicable in every situation.

Is using step size variation always possible? If we consider a \emph{time-inhomogeneous Poisson process noise}, which would generally take the form
%%
\begin{align}
S^{i}_{{\sf t}+1}(X_{0:{\sf t}},z,{\sf t}) &= [N_{\lambda ({\sf t}+1)}]^i_{{\sf t}+1}-[N_{\lambda ({\sf t})}]^i_{\sf t}\,,
\end{align}
%%
the rate $\lambda ({\sf t})$ has become a deterministically-varying function in time. In this instance, it likely not be accurate to simulate this process by drawing exponential intervals with a mean of $1/\lambda ({\sf t})$ because this mean could have changed by the end of the interval which was drawn. An alternative approach (which is more generally capable of simulating jump processes but is an approximation) first uses a small time interval $\tau$ such that the most likely thing to happen in this period is nothing, and then the probability of the event occuring is simply given by
%%
\begin{align}
p({\sf event}) &= \frac{\lambda ({\sf t})}{\lambda ({\sf t}) + \frac{1}{\tau}} \label{eq:rejection}\,.
\end{align}
%%
This idea can be applied to phenomena with an arbitrary number of events and works well as a generalised approach to event-based simulation, though its main limitation is worth remembering; in order to make the approximation good, $\tau$ often must be quite small and hence our simulator must churn through a lot of steps. From now on we'll refer to this well-known technique as the \emph{rejection method}. Fig.~\ref{fig:inhomogeneous-poisson} may also help to understand this concept from the programmer's perspective.

\begin{figure}[h]
\centering
\includegraphics[width=9cm]{images/chapter-1-inhomogeneous-poisson.drawio.png}
\caption{Schematic of code for an inhomogeneous Poisson process.}
\label{fig:inhomogeneous-poisson}
\end{figure}

There are a few extensions to the simple Poisson process that introduce additional stochastic processes. \emph{Cox (doubly-stochastic) processes}, for instance, are basically where we replace the time-dependent rate $\lambda ({\sf t})$ with independent samples from some other stochastic process $\Lambda ({\sf t})$. For example, a Neyman-Scott process~\cite{neyman1958statistical} can be mapped as a special case of this because it uses a Poisson process on top of another Poisson process to create maps of spatially-distributed points. In our formalism, a two-state implementation of the Cox process noise would look like
%%
\begin{align}
S^{0}_{{\sf t}+1}(X_{0:{\sf t}},z,{\sf t}) &= \Lambda ({\sf t}+1) \\
S^{1}_{{\sf t}+1}(X_{0:{\sf t}},z,{\sf t}) &= [N_{S^{0}_{{\sf t}+1}}]^i_{{\sf t}+1}-[N_{S^{0}_{{\sf t}}}]^i_{\sf t}\,.
\end{align}
%%
This process could be simulated using the pseudocode we wrote for the time-inhomogeneous Poisson process previously --- where we would just replace \texttt{EventRateLambdaFunction} with a function that generates the stochastic rate $\Lambda ({\sf t})$.

Another extension is \emph{compound Poisson process noise}, where it's the count values $[N_{\lambda}]^i_{\sf t}$ which are replaced by independent samples $[J_{\lambda}]^i_{\sf t}$ from another probability distribution, i.e.,
%%
\begin{align}
S^{i}_{{\sf t}+1}(X_{0:{\sf t}},z,{\sf t}) &= [J_{\lambda}]^i_{{\sf t}+1}-[J_{\lambda}]^i_{\sf t}\,.
\end{align}
%%
Note that the rejection method of Eq.~(\ref{eq:rejection}) can be employed effectively to simulate any of these extensions as long as a sufficiently small $\tau$ is chosen. Once again, the pseudocode we wrote previously would be sufficient to simulate this process with one tweak: add into the \texttt{DrawNewEventIncrement} function the calling of a function which generates the $[J_{\lambda}]^i_{\sf t}$ samples and output these if the event occurs.

All of the examples we have discussed so far are Markovian. Given that we have explicitly constructed the formalism to handle non-Markovian phenomena as well, it would be worthwhile going some examples of this kind of process too. \emph{Self-exciting process noises} would generally take the form
%%
\begin{align}
S^{0}_{{\sf t}+1}(X_{0:{\sf t}},z,{\sf t}) &= {\cal I}_{{\sf t}+1} (X_{0:{\sf t}},z,{\sf t}) \\
S^{1}_{{\sf t}+1}(X_{0:{\sf t}},z,{\sf t}) &= [N_{S^{0}_{{\sf t}+1}}]^i_{{\sf t}+1}-[N_{S^{0}_{{\sf t}}}]^i_{\sf t} \,,
\end{align}
%%
where the stochastic rate ${\cal I}_{{\sf t}+1} (X_{0:{\sf t}},z,{\sf t})$ now depends on the history explicitly. Amongst other potential inputs we can see, e.g., Hawkes processes~\cite{hawkes1971spectra} as an example of above by substituting 
%%
\begin{align}
{\cal I}_{{\sf t}+1} (X_{0:{\sf t}},z,{\sf t}) &= \mu + \sum^{{\sf t}}_{{\sf t}'=0}\gamma [t({\sf t})-t({\sf t}')]S^{1}_{{\sf t}'} \,,
\end{align}
%%
where $\gamma$ is the `exciting kernel' and $\mu$ is some constant background rate. In order to simulate a Hawkes process using our formalism, the pseudocode would be something like Fig.~\ref{fig:hawkes-process}.

\begin{figure}[h]
\centering
\includegraphics[width=9cm]{images/chapter-1-hawkes-process.drawio.png}
\caption{Schematic of code for a Hawkes process.}
\label{fig:hawkes-process}
\end{figure}

Note that this idea of integration kernels could also be applied back to our Wiener process. For example, another type of non-Markovian phenomenon that frequently arises across physical and life systems integrates the Wiener process history like so
%%
\begin{align}
S^{0}_{{\sf t}+1}(X_{0:{\sf t}},z,{\sf t}) &= W^0_{{\sf t}+1}-W^0_{\sf t}\\
S^{1}_{{\sf t}+1}(X_{0:{\sf t}},z,{\sf t}) &= u\sum^{{\sf t}}_{{\sf t}'=0}e^{-u[t({\sf t})-t({\sf t}')]} S^{0}_{{\sf t}'}\,,
\end{align}
%%
where $u$ is inversely proportional to the length of memory in continuous time.

\section{\sffamily Software design}

So we've proposed a computational formalism and then studied it in more detail to demonstrate that it can cope with a variety of different stochastic phenomena. Now we're ready to summarise what we want the stochadex software package to be able to do. But what's so complicated about Eq.~(\ref{eq:x-step-def})? Can't we just implement an iterative algorithm with a single function? It's true that the fundamental concept is very straightforward, but as we'll discuss in due course; the stochadex needs to have a lot of configurable features so that it's applicable in different situations. Ideally, the stochadex sampler should be designed to try and maintain a balance between performance and flexibility of utilisation.

If we begin with the obvious first set of criteria; we want to be able to freely configure the iteration function $F$ of Eq.~(\ref{eq:x-step-def}) and the timestep function $t$ of Eq.~(\ref{eq:t-steps-sum}) so that any process we want can be described. The point at which a simulation stops can also depend on some algorithm termination condition which the user should be able to specify up-front.

\begin{figure}[h]
\centering
\includegraphics[width=13cm]{images/chapter-1-stochadex-data-types.drawio.png}
\caption{A relational summary of the configuration data types in the stochadex.}
\label{fig:data-types-design}
\end{figure}

Once the user has written the code to create these functions for the stochadex, we want to then be able to recall them in future only with configuration files while maintaining the possibility of changing their simulation run parameters. This flexibility should facilitate our uses for the simulation later in the book, and from this perspective it also makes sense that the parameters should include the random seed and initial state value.

The state history matrix $X$ should be configurable in terms of its number of rows --- what we'll call the `state width' --- and its number of columns --- what we'll call the `state history depth'. If we were to keep increasing the state width up to millions of elements or more, it's likely that on most machines the algorithm performance would grind to a halt when trying to iterate over the resulting $X$ within a single thread. Hence, before the algorithm or its performance in any more detail, we can pre-empt the requirement that $X$ should represented in computer memory by a set of partitioned matrices which are all capable of communicating to one-another downstream. In this paradigm, we'd like the user to be able to configure which state partitions are able to communicate with each other without having to write any new code.

For convenience, it seems sensible to also make the outputs from stochadex runs configurable. A user should be able to change the form of output that they want through, e.g., some specified function of $X$ at the time of outputting data. The times that the stochadex should output this data can also be decided by some user-specified condition so that the frequency of output is fully configurable as well. 

In summary, we've put together a schematic of configuration data types and their relationships in Fig.~\ref{fig:data-types-design}. In this diagram there is some indication of the data type that we propose to store each piece information in (in Go syntax), and the diagram as a whole should serve as a useful guide to the basic structure of configuration files for the stochadex.

It's clear that in order to simulate Eq.~(\ref{eq:x-step-def}), we need an interative algorithm which reapplies a user-specified function to the continually-updated history. But let's now return to the point we made earlier about how the performance of such an algorithm will depend on the size of the state history matrix $X$. The key bit of the algorithm design that isn't so straightforward is: how do we sucessfully split this state history up into separate partitions in memory while still enabling them to communicate effectively with each other? Other generalised simulation frameworks --- such as SimPy~\cite{simpy}, StoSpa~\cite{stospa} and FLAME GPU~\cite{flamegpu} --- have all approached this problem in different ways, and with different software architectures. 

In Fig.~\ref{fig:loop-design} we've illustrated what a loop involving separate state partitions looks like in the stochadex simulator. Each partition is handled by concurrently running execution threads of the same process, while a separate process may be used to handle the outputs from the algorithm. As the diagram shows, the main sequence of each loop iteration follows the pattern: 
%%
\begin{enumerate}
\item{The \texttt{PartitionCoordinator} requests more iterations from each state partition by sending an \texttt{IteratorInputMessage} to a concurrently running goroutine.}
\item{The \texttt{StateIterator} in each goroutine executes the iteration and stores the resulting state update in a variable.}
\item{Once all of the iterations have been completed, the \texttt{PartitionCoordinator} then requests each goroutine to update its relevant partition of the state history by sending another \texttt{IteratorInputMessage} to each.}
\end{enumerate}
%%
This pattern ensures that no partition has access to values in the state history which are out of sync with its current state in time, and hence prevents anachronisms from occuring in the overall state iteration. 

\begin{figure}[h]
\centering
\includegraphics[width=13cm]{images/chapter-1-stochadex-loop.drawio.png}
\caption{Schematic for a step of the stochadex simulation algorithm.}
\label{fig:loop-design}
\end{figure}

It's also worth noting that while Fig.~\ref{fig:loop-design} illustrates only a single process; it's obviously true that we may run many of these whole diagrams at once to parallelise generating independent realisations of the simulation, if necessary.

As we stated at the beginning of this chapter: the full implementation of the stochadex can be found on GitHub by following this link: \href{https://github.com/umbralcalc/stochadex}{https://github.com/umbralcalc/stochadex}. Users can build the main binary executable of this repository and determine what configuration of the stochadex they would like to have through config at runtime (one can infer these configurations from Fig.~\ref{fig:data-types-design}). As Go is a statically typed language, this level of flexibility has been achieved using code templating proceeding runtime build and execution via \texttt{go run} `under-the-hood'. Users who find this particular execution pattern undesirable can also use all of the stochadex types, tools and methods as part of a standard library import.

In order to debug the simulation code and gain a more intuitive understanding of the outputs from a model as it is being developed, we have also written a lightweight frontend dashboard React~\cite{react} app in TypeScript to visualise any stochadex simulation as it is running. This dashboard can be launched by passing config at runtime to the main stochadex executable, and we have illustrated how all this fits together in a flowchart shown in Fig.~\ref{fig:stochadex-main}.

\begin{figure}[h]
\centering
\includegraphics[width=9cm]{images/chapter-1-stochadex-main.drawio.png}
\caption{A diagram of the main stochadex binary executable.}
\label{fig:stochadex-main}
\end{figure}



\chapter{\sffamily Numerical time evolution of probabilities}

{\bfseries\sffamily Concept.} To extend the formalism that we developed in previous chapters to... For the mathematically-inclined, this chapter will take a detailed look at how our formalism can be extended to focus on probabilistic modelling. For the programmers, the software described in this chapter lives in the public Git repository: \href{https://github.com/umbralcalc/dennm-torch}{https://github.com/umbralcalc/dennm-torch}.

\section{\sffamily Probabilistic formalism}

In this section we will return to the stochadex formalism that we introduced in the first chapter of this book. As we discussed at that point; this formalism is appropriate for sampling from nearly every stochastic phenomenon that one can think of. However, when trying robustly assess how far a model is from accurately describing a set of real-world data, trying to use only generated samples of the model process can be diffcult. Instead, in this section, we are going to extend this formalism to look at how probability theory can help with this data comparison problem in a systematic way.

\begin{figure}[h]
\centering
\includegraphics[width=8cm]{images/chapter-3-master-eq-graph.drawio.png}
\caption{Graph representation of Eqs.~(\ref{eq:master-x-cont}) and~(\ref{eq:master-x-cont-latest-row}).}
\label{fig:master-eqn}
\end{figure} 

So, how do we begin? In the first chapter, we defined the general stochastic process with the formula $X^{i}_{{\sf t}+1} = F^{i}_{{\sf t}+1}(X_{0:{\sf t}},z,{\sf t})$. This equation also has an implicit \emph{master equation} associated to it that fully describes the time evolution of the \emph{probability density function} $P_{{\sf t}+1}(X\vert z)$ of $X_{0:{\sf t}+1}=X$ given that the parameters of the process are $z$. This can be written as
%%
\begin{align}
P_{{\sf t}+1}(X\vert z) &= P_{{\sf t}}(X'\vert z) P_{({\sf t}+1){\sf t}}(x\vert X',z) \label{eq:master-x-cont}\,,
\end{align}
%%
where for the time being we are assuming the state space is continuous in each of the matrix elements and $P_{({\sf t}+1){\sf t}}(x\vert X',z)$ is the conditional probability that $X_{{\sf t}+1}=x$ given that $X_{0:{\sf t}}=X'$ at time ${\sf t}$ and the parameters of the process are $z$.

If we wanted to just look at the distribution over the latest row $X_{{\sf t}+1}=x$, we could achieve this through marginalisation over all of the previous matrix rows in Eq.~(\ref{eq:master-x-cont}) like this
%%
\begin{align}
P_{{\sf t}+1}(x\vert z) = \int_{\Omega_{{\sf t}}}{\rm d}X' P_{{\sf t}+1}(X\vert z) &= \int_{\Omega_{{\sf t}}}{\rm d}X' P_{{\sf t}}(X'\vert z) P_{({\sf t}+1){\sf t}}(x\vert X',z) \label{eq:master-x-cont-latest-row} \,.
\end{align}
%%
But what is $\Omega_{\sf t}$? You can think of this as just the domain of possible matrix $X'$ inputs into the integral which will depend on the specific stochastic process we are looking at. 

The symbol ${\rm d}X'$ in Eq.~(\ref{eq:master-x-cont-latest-row}) is our shorthand notation throughout the book for taking a sum of sub-domain integrals over each matrix row; where each row measure is a Cartesian product of $n$ elements (a Lebesgue measure), i.e.,
%%
\begin{align}
\int_{\Omega_{{\sf t}}}{\rm d}X' = \frac{1}{{\sf t}}\sum_{{\sf t}'=0}^{{\sf t}} \int_{\omega_{{\sf t}'}}{\rm d}^nx' = \frac{1}{{\sf t}}\sum_{{\sf t}'=0}^{{\sf t}} \int_{\omega_{{\sf t}'}}\prod_{i=0}^n{\rm d}(x')^i \,,
\end{align}
%%
where lowercase $x, x', \dots$ values will always refer to individual rows within the state matrices. Note that $1/{\sf t}$ here is a normalisation factor --- this just normalises the sum of all probabilities to 1 given that there is a sum over ${\sf t}'$. Note also that, if the process is defined over continuous time, we would need to replace 
%%
\begin{align}
\frac{1}{{\sf t}}\sum_{{\sf t}'=0}^{{\sf t}} \rightarrow \frac{1}{t({\sf t})}\sum_{{\sf t}'=0}^{{\sf t}}\delta t({\sf t}') \,.
\end{align}
%%

To try and understand what Eqs.~(\ref{eq:master-x-cont}) and~(\ref{eq:master-x-cont-latest-row}) are saying, we find it's helpful to think of an iterative relationship between probabilities; each of which is connected by their relative conditional probabilities. This kind of thinking is also illustrated in Fig.~\ref{fig:master-eqn}.

Without loss of generality, we can relate the latest probabilities to those from deeper into the past by chaining conditional probabilities together in a non-Markovian equivalent of the Chapman-Kolmogorov equation
%%
\begin{align}
P_{{\sf t}+1}(x\vert z) &= \int_{\Omega_{{\sf t}-1}}{\rm d}X''P_{{\sf t}-1}(X''\vert z)\int_{\omega_{{\sf t}}}{\rm d}^nx' P_{{\sf t}({\sf t}-1)}(x'\vert X'',z)P_{({\sf t}+1){\sf t}}(x\vert X',z) \nonumber \\
&= \int_{\Omega_{{\sf t}-2}}{\rm d}X'''P_{{\sf t}-2}(X'''\vert z)\int_{\omega_{{\sf t}-1}}{\rm d}^nx''P_{({\sf t}-1)({\sf t}-2)}(x''\vert X''',z) \nonumber \\
&\qquad \qquad \qquad \quad \times\int_{\omega_{{\sf t}}}{\rm d}^nx' P_{{\sf t}({\sf t}-1)}(x'\vert X'',z)P_{({\sf t}+1){\sf t}}(x\vert X',z) \nonumber \\
&= \dots \nonumber \\
&= \int_{\Omega_{{\sf t}-{\sf s}}}{\rm d}X'''P_{{\sf t}-{\sf s}}(X'''\vert z)\prod_{{\sf s}'=0}^{{\sf s}-1} \bigg\lbrace \int_{\omega_{{\sf t}-{\sf s}'}}{\rm d}^nx' P_{({\sf t}-{\sf s}')({\sf t}-{\sf s}'-1)}(x'\vert X'',z) \bigg\rbrace P_{({\sf t}+1){\sf t}}(x\vert X',z) \,.
\end{align}
%%

Depending on the temporal correlation structure of the process, the conditional probabilities can be factorised. For example, processes with second or third-order temporal correlations would be described by the following expressions
%%
\begin{align}
P_{({\sf t}+1){\sf t}}(x\vert X',z) &= \frac{1}{{\sf t}}\sum_{{\sf t}'=0}^{{\sf t}}\int_{\omega_{{\sf t}'}}{\rm d}^nx' P_{{\sf t}'}(x'\vert z)P_{({\sf t}+1){\sf t}'}(x\vert x',z) \\
P_{({\sf t}+1){\sf t}}(x\vert X',z) &= \frac{1}{{\sf t}}\sum_{{\sf t}'=0}^{{\sf t}}\frac{1}{{\sf t}'}\sum_{{\sf t}'=0}^{{\sf t}'}\int_{\omega_{{\sf t}'}}{\rm d}^nx'\int_{\omega_{{\sf t}''}}{\rm d}^nx'' P_{{\sf t}''}(x''\vert z) P_{{\sf t}'{\sf t}''}(x'\vert x'',z)P_{({\sf t}+1){\sf t}'{\sf t}''}(x\vert x',x'',z) \,.
\end{align}
%%

Let's imagine that $x$ is just a scalar (as opposed to a row vector) for simplicity in the expressions. We can then discretise the 1D space over $x$ into separate $i$-labelled regions such that $[P]^i_{{\sf t}+1} - [P]^i_{{\sf t}} = {\cal J}^i_{{\sf t}+1}$, where the probability current ${\cal J}^i_{{\sf t}+1}$ for the factorised processes above would be defined as
%%
\begin{align}
{\cal J}^i_{{\sf t}+1} &= - [P]^i_{{\sf t}} + \frac{1}{{\sf t}}\sum^{{\sf t}}_{{\sf t}'=0}\sum_{i'=0}^N\Delta x[P]^{i'}_{{\sf t}'}[P]^{ii'}_{({\sf t}+1){\sf t}'} \\
{\cal J}^i_{{\sf t}+1} &= - [P]^i_{{\sf t}} + \frac{1}{{\sf t}}\sum^{{\sf t}}_{{\sf t}'=1}\frac{1}{{\sf t}'-1}\sum^{{\sf t}'-1}_{{\sf t}''=0}\sum_{i'=0}^N\sum_{i''=0}^N\Delta x^2[P]^{i''}_{{\sf t}''}[P]^{i'i''}_{{\sf t}'{\sf t}''}[P]^{ii'i''}_{({\sf t}+1){\sf t}'{\sf t}''}\,.
\end{align}
%%

The $[P]^{ii'i''}_{({\sf t}+1){\sf t}'{\sf t}''}$ tensor, in particular, will have $N^3{\sf t}({\sf t}^2-1)$ elements. Note that the third-order temporal correlations can be evolved by identifying the pairwise conditional probabilities as time-dependent state variables and evolving them according to the following relation
%%
\begin{align}
[P]^{ii''}_{({\sf t}+1){\sf t}''} &= \frac{1}{{\sf t}}\sum^{{\sf t}}_{{\sf t}'=1}\sum_{i'=0}^N\Delta x[P]^{i'i''}_{{\sf t}'{\sf t}''}[P]^{ii'i''}_{({\sf t}+1){\sf t}'{\sf t}''}\,.
\end{align}
%%

What other classes of process can be described by Eqs.~(\ref{eq:master-x-cont}) and~(\ref{eq:master-x-cont-latest-row})? For Markovian phenomena, the equations no longer depend on timesteps older than the immediately previous one, hence Eq.~(\ref{eq:master-x-cont-latest-row}) reduces to just
%%
\begin{align}
P_{{\sf t}+1}(x\vert z) &= \int_{\omega_{\sf t}}{\rm d}^nx' \, P_{\sf t}(x'\vert z) P_{({\sf t}+1){\sf t}}(x\vert x',z) \label{eq:master-x-cont-markov} \,.
\end{align}
%%
An analog of Eq.~(\ref{eq:master-x-cont-latest-row}) exists for discrete state spaces as well. We just need to replace the integral with a sum and the schematic would look something like this
%%
\begin{align}
P_{{\sf t}+1}(x\vert z) &= \sum_{\Omega_{{\sf t}}} P_{{\sf t}}(X'\vert z) P_{({\sf t}+1){\sf t}}(x \vert X', z) \label{eq:master-x-disc} \,,
\end{align}
%%
where we note that the $P$'s in the expression above all now refer to \emph{probability mass functions}.

In this section we looked into how the mathematical formalism used in the stochadex could be extended with probability theory.

    

\chapter{\sffamily Learning from ants on curved surfaces}

{\bfseries\sffamily Concept.} The idea here is 

\section{\sffamily Diffusive limits for ant interactions}

\chapter{\sffamily Learning from ants on curved surfaces}

{\bfseries\sffamily Concept.} The idea here is 

\section{\sffamily Diffusive limits for ant interactions}

\chapter{\sffamily Optimising interactions with any system}

{\bfseries\sffamily Concept.} To design and build software which enables the optimisation of automated control objectives over stochastic phenomena of any kind. The theory in this chapter will overlap significantly with that of Reinforcement Learning (RL), however, in contrast to more standard RL approaches, we shall be relying on all of the work from previous parts of this book to help agents characterise, measure and learn from their environment. The software which implements our generalised control optimisation algorithm will be implemented as an extension to the learnadex. For the mathematically-inclined, this chapter will cover how we formalise model-based automated control optimisation within the frameworks that we have already introduced in this book. For the programmers, the public Git repository for the code described in this chapter can be found here: \href{https://github.com/worldsoop/worldsoop}{https://github.com/worldsoop/worldsoop}.

\section{\sffamily Formalising general interactions}

Let's start by considering how we might adapt the mathematical formalism we have been using so far to be able to take actions which can manipulate the state at each timestep. Using the mathematical notation that we inherited from the stochadex, we may extend the formula for updating the state history matrix $X_{0:{\sf t}}\rightarrow X_{0:{\sf t}+1}$ to include a new layer of possible interactions which is facilitated by a new vector-valued `take action' function $G_{{\sf t}}$. In doing so we shall be defining the domain of an acting entity in the stochastic process environment --- which we shall hereafter refer to as simply the `agent'.

During a timestep over which actions are performed by the agent, the stochadex state update formula can be extended to include interactions by composition with the original state update function like so
%%
\begin{align}
X_{{\sf t}+1}^i &= G^i_{{\sf t}+1}[F_{{\sf t}+1}(X_{0:{\sf t}}, z, {\sf t}), A_{{\sf t}+1}] = {\cal F}^i_{{\sf t}+1}(X_{0:{\sf t}}, z, A_{{\sf t}+1}, {\sf t}) \label{eq:generalised-state-actions} \,,
\end{align}
%%
where we have also introduced the concept of the `actions' performed $A_{{\sf t}+1}$ on the system; some vector of parameters which define what actions are taken at timestep ${\sf t}+1$. The code for the new iteration formula would look something like Fig.~\ref{fig:iterations-with-actions}.

\begin{figure}[h]
\centering
\includegraphics[width=11cm]{images/chapter-5-iterations-with-actions.drawio.png}
\caption{Code schematic of Eq.~(\ref{eq:generalised-state-actions}).}
\label{fig:iterations-with-actions}
\end{figure}

So far, Eq.~(\ref{eq:generalised-state-actions}) on its own will allow the agent to take actions that are scheduled up front through some fixed process or perhaps through user interaction via a game interface. So what's next? In order to start creating algorithms which will act on the system state for us, we need to develop a formalism which `closes the loop' by feeding information back from the stochastic process to the agent's decision-making algorithm.

If we use $A_{0:{\sf t}+1}$ a referring to the matrix of historically-taken actions which up to time ${\sf t}+1$, we can build up a more generalised, non-Markovian picture of automated interactions with the system which matches the notation we are already using for $X_{0:{\sf t}+1}$. Let us now define a Non-Markovian Decision Process (NMDP) as a probabilistic model which draws an actions matrix $A_{0:{\sf t}+1}=A$ from a `policy' distribution $\Pi_{({\sf t}+1){\sf t}}(A\vert X,\theta)$ given $X_{0:{\sf t}}=X$ and a new vector of parameters which fully specify the automated interations. Using the probabilistic notation from the previous part of the book, the joint probability that $X_{0:{\sf t}+1}=X$ and $A_{0:{\sf t}+1}=A$ at time ${\sf t}+1$ is
%%
\begin{align}
P_{{\sf t}+1}(X,A\vert z, \theta ) &= P_{{\sf t}}(X'\vert z,\theta ) \, \Pi_{({\sf t}+1){\sf t}}(A\vert X',\theta)P_{({\sf t}+1){\sf t}}(x\vert X',z,A) \label{eq:joint-prob-x-and-a} \,,
\end{align}
%%
where we recall that $P_{({\sf t}+1){\sf t}}(x\vert X',z,A)$ is the conditional probability of $X_{{\sf t}+1}=x$ given $X_{0:{\sf t}}=
X'$ and $z$ that we have encountered before, but it now requires $A_{0:{\sf t}+1}=A$ as another given input. We have illustrated Eq.~(\ref{eq:generalised-state-actions}) and how it relates to the policy distribution of Eq.~(\ref{eq:joint-prob-x-and-a}) with a new graph representation in Fig.~\ref{fig:fundamental-loop-with-actions}.

\begin{figure}[h]
\centering
\includegraphics[width=11cm]{images/chapter-5-fundamental-loop-with-actions.drawio.png}
\caption{Graph representation of Eq.~(\ref{eq:generalised-state-actions}) with the policy distribution of Eq.~(\ref{eq:joint-prob-x-and-a}).}
\label{fig:fundamental-loop-with-actions}
\end{figure}

For additional clarity, let's take a moment to think about what $\Pi_{({\sf t}+1){\sf t}}(A\vert X,\theta)$ represents and how generally descriptive it can be. If an agent is acting under and entirely deterministic policy, then the policy distribution may be simplified to a direct function mapping which is parameterised by $\theta$. At the other extreme, the distribution may also describe a fully stochastic policy where actions are drawn randomly in time. If we combine this consideration of noise with the observation that policies described by a distribution $\Pi_{({\sf t}+1){\sf t}}(A\vert X,\theta)$ permit a memory of past actions and states, it's easy to see that this structure can be used in a wide variety of different use cases.

By marginalising over Eq.~(\ref{eq:joint-prob-x-and-a}) we find an updated probabilistic iteration formula for the stochastic process state which now takes the influence of agent actions into account
%%
\begin{align}
P_{{\sf t}+1}(X\vert z,\theta ) &= \int_{\Xi_{{\sf t}+1}}{\rm d}A \, P_{{\sf t}}(X'\vert z,\theta ) \, \Pi_{({\sf t}+1){\sf t}}(A\vert X',\theta)P_{({\sf t}+1){\sf t}}(x\vert X',z,A)  \,.
\end{align}
%%
This relationship will be very useful in the last part of this book when we begin to look at optimising control algorithms.

What are the main categories of action which are possible in the rows of $A$? Since the NMDP described by $\Pi_{({\sf t}+1){\sf t}}(A\vert X',\theta)$ is just another form of stochastic process, the main categories of action will fall into the same as those we covered in defining the stochadex formalism. The first, and perhaps most obvious, category would probably where the actions are defined in a continuous space and are continuously applied on every timestep. Some examples of these `continuously-acting' decision processes include controlling the temperature of chemical reactions~\cite{beeler2023chemgymrl} (such as those in a brewery), spacecraft control~\cite{tipaldi2022reinforcement} and guidance systems,  as well as the driving of autonomous vehicles~\cite{kiran2021deep}. Within a kind of subset of the continuously-acting category; we can also find the event-based acting decision processes (where actions are not necessarily taken every timestep), e.g. controlling traffic through signal timings~\cite{garg2018deep}, managing disease spread through treatment intervals~\cite{ohi2020exploring} and automated trading on stock markets~\cite{meng2019reinforcement}.

Many of the examples we have given above have continuous action spaces, but we might also consider classes of decision processes where actions are defined discretely. Examples of these include the famous multi-armed bandit problem~\cite{gittins2011multi} (like choosing between website layouts for E-commerce~\cite{liu2021map}), managing a sports team through player substitutions, sensor measurement scheduling~\cite{leong2020deep} and the sequential design prioritisation of large-scale scientific experiments~\cite{blau2022optimizing}.

\section{\sffamily States, actions and attributing rewards}

In the previous parts of this book we laid out the concept for a generalised framework to simulate and learn stochastic phenomena continually as data is received. Given that we have also introduced a framework for the automated control of these phenomena, we have all the ingredients we need to create optimal decision-making algorithms. The key question to answer then, is: \emph{optimal with respect to what objective?}

The objective of an automated control algorithm could take many forms depending on the specific context. Since there is no loss in generality in doing so, it seems natural to follow the naming convention used by Markov Decision Processes (MDP)~\cite{bertsekas2011dynamic,sutton2018reinforcement} by referring to the objective outcome of an action at a particular point in time as having a `reward' value $r$. Since the relationship between reward, actions and states may be stochastic, it makes sense to relate the reward outcome $r$ given a state history $X$ and action history $A$ at timestep ${\sf t}+1$ through the probability distribution $P_{{\sf t}+1}(r\vert X,A)$. Hence, generally, this reward signal is non-Markovian --- as is the case in many real-world problems~\cite{gaon2020reinforcement}. 

We can use the reward probability distribution to derive a joint distribution over both state history $X'$ and reward $r$ at timestep ${\sf t}+1$ like so
%%
\begin{align}
P_{({\sf t}+1){\sf t}}(r,x'\vert X, z,\theta) &= P_{{\sf t}+1}(r\vert X',A)\Pi_{({\sf t}+1){\sf t}}(A\vert X,\theta)P_{({\sf t}+1){\sf t}}(x'\vert X,z,A) \label{eq:joint-x-and-r}\,.
\end{align}
%%
In this expression, let's recall that we are using the policy distribution $\Pi_{({\sf t}+1){\sf t}}(A\vert X,\theta)$ for agent interactions and the fundamental state update conditional probability for the underlying stochastic process $P_{({\sf t}+1){\sf t}}(x'\vert X,z,A)$.

Note that in most use cases, the state of real-world phenomena cannot be measured perfectly. So to enable any agent trained on simulated phenomena to potentially act in the real world, we will need to include a measurement process as part of the information retrieval step. This is the part where we can leverage our work in a previous chapter which develops an online learning system for stochastic process models. But we're jumping ahead with this thinking and will return to this point later on.

Using Eq.~(\ref{eq:joint-x-and-r}), we can now define a `state value function' $V_{{\sf t}}$ at timestep ${\sf t}$ which is the expected $\gamma$-discounted future reward given the current state history $X$ and the other parameters like this\footnote{The discount factor in continuous time could also be explicitly dependent on the stepsize such that we would replace the discount factor in Eq.~(\ref{eq:state-value-discounted-return}) with
$$
\gamma^{{\sf t}'-{\sf t}} \longrightarrow \frac{1}{\gamma [\delta t({\sf t}+1)]}\prod_{{\sf t}''={\sf t}}^{{\sf t}'} \gamma [\delta t({\sf t}''+1)] \,.
$$}
%%
\begin{align}
V_{{\sf t}}(X,z,\theta) &= {\rm E}_{{\sf t}}({\sf Discounted \,Return}\vert X, z, \theta ) \nonumber \\
&= \sum_{{\sf t}'={\sf t}}^{\infty} \int_{\omega_{{\sf t}'+1}}{\rm d}^nx'\int_{\rho_{{\sf t}'+1}} {\rm d}r \,r\, \gamma^{{\sf t}'-{\sf t}}\prod_{{\sf t}''={\sf t}}^{{\sf t}'}P_{({\sf t}''+1){\sf t}''}(r,x'\vert X, z,\theta) \label{eq:state-value-discounted-return}\,,
\end{align}
%%
where $0 < \gamma < 1$. The idea behind this discount factor $\gamma$ is to decrease the contribution of rewards to the optimisation objective (often called the `expected discounted return' in RL) more and more as the prediction increases into the future. Note also that the state value function is inherently recursively defined, such that
%%
\begin{align}
V_{{\sf t}}(X,z,\theta) &= \int_{\omega_{{\sf t}+1}}{\rm d}^nx\int_{\rho_{{\sf t}+1}} {\rm d}r \, P_{({\sf t}+1){\sf t}}(r,x'\vert X, z,\theta)\big[ r+\gamma V_{{\sf t}+1}(X',z,\theta)\big] \,,
\end{align}
%%
and the optimal $\theta$ can hence be derived from
%%
\begin{align}
\theta^*_{{\sf t}}(X,z) &= \underset{\theta}{{\rm argmax}} \big[ V_{{\sf t}}(X,z,\theta)\big] \,.
\end{align}
%%
By deriving the optimal policy in terms of the parameters $\theta^*_{{\sf t}}(X,z)$, the optimal state value function and policy distribution can therefore be derived from
%%
\begin{align}
V^*_{{\sf t}}(X,z) &= V_{{\sf t}}[X,z,\theta^*_{{\sf t}}(X,z)] \\
\Pi^*_{({\sf t}+1){\sf t}}(A\vert X,z) &= \Pi_{({\sf t}+1){\sf t}}[A\vert X,\theta^*_{{\sf t}}(X,z)] \,.
\end{align}
%%

Note that the type of decision process optimisation which we have introduced above differs from standard RL methodology. In the more conventional `model-free' RL approaches, the state-action value function 
%%
\begin{align}
Q_{{\sf t}}(X,A,z)={\rm E}_{{\sf t}}({\sf Discounted \,Return}\vert X,A,z) \,,
\end{align}
%%
would be used to evaluate the optimal policy instead of the state value function $V_{{\sf t}}(X,z,\theta )$ that we are using above. We are able to use the latter here because the simulation model gives us explicit knowledge of the $P_{({\sf t}+1){\sf t}}(x'\vert X,z,A)$ distribution which is utilised by Eq.~(\ref{eq:joint-x-and-r}). When this model is not known, the state-action value function $Q_{{\sf t}}(X,A,z)$ must be learned explicitly through sample estimation from the measured state and experienced outcomes of actions taken by the agent.

When an agent takes an action to measure the state of the system (or when it is given measurements without needing to take action) there will typically be some uncertainty in how the history of measured real-world data $Y$ maps to the latent states of the system $X$ and its parameters $z$ at time ${\sf t}+1$. It is natural, then, to represent this uncertainty with a posterior probability distribution ${\cal P}_{{\sf t}+1}(X,z\vert Y)$ as we did in the previous chapters of this book.

\section{\sffamily Algorithm designs}

\textcolor{red}{
Follow-up this bit with the model-based approach that we're going to take in this book.
\begin{itemize}
\item{Introduce broad concept of dynamic programming --- partitioning a optimal global control into smaller optimal control segments/iterations.}
\item{Talk about the utility of the model-based online learning approach in the case of partially observed systems~\cite{aastrom1965optimal}.}
\item{Look into the overlaps with this approach and Thompson sampling for exploration --- discuss here.}
\item{Looking at a stochastic policy iteration algorithm here combined with Monte Carlo rollouts.}
\item{The value learning can be facilitated in software using a predictive model which is able to roll forecast rewards forward in time in a Monte Carlo fashion up to a window from a certain point given an input prior distribution of policies.}
\item{This input prior distribution of policies can itself be optimised by maximising expected discounted utility in a Bayesian design framework. Draw parallels.}
\end{itemize}
}

\section{\sffamily Software design}

\begin{figure}[h]
\centering
\includegraphics[width=12cm]{images/chapter-5-high-level-dependencies.drawio.png}
\caption{Diagram illustrating the high-level layer inter-dependencies of the Python API.}
\label{fig:high-level-api-dependencies}
\end{figure}

\part*{{\sffamily Part 2}}

\chapter{\sffamily Controlling parasitic infections}

{\bfseries\sffamily Concept.} The idea here is to limit the spread of some abstract spatial parasitic infections through the correct time-dependent resource allocation.

\section{\sffamily Adapting the probabilistic formalism}

Let's by returning to the probabilistic formalism that we introduced earlier and noting that the covariance matrix estimate with elements $C^{ij}_{{\sf t}+1}(z)$ represents a matrix that could get very large, depending on the problem. For example; if we encoded the state of a 2-dimensional spatial field of values into the elements $X^i_{\sf t}$, the number of elements in the covariance matrix $C^{ij}_{{\sf t}+1}(z)$ would scale as $4N^2$ --- where $N$ here is the number of spatial points we wanted to encode. 

One solution to this scaling problem is to exploit the fact that, in many spatial processes, the proximity of points can strongly determine how correlated they are. Hence, for pairwise distances further than some threshold, the covariance matrix elements should tend towards 0. If we were to place points along the diagonal of $C^{ij}_{{\sf t}+1}(z)$ in order of how close they are to each other, this threshold would then be represented as a \emph{banded matrix}. We have illustrated such a matrix in Fig.~\ref{fig:banded-matrix} in which the `bandwidth' is defined as the number of diagonals one needs to traverse from the main diagonal before encountering a diagonal of 0s.

\begin{figure}[h]
\centering
\includegraphics[width=9cm]{images/chapter-7-banded-matrix.drawio.png}
\caption{An illustration of a banded covariance matrix with a bandwidth of 2.}
\label{fig:banded-matrix}
\end{figure}

\textcolor{red}{
\begin{itemize}
\item{At some point it might be sensible to move into the Fourier domain here --- at least for derivations and calculations. Probably more intuitive for the reader to keep it mostly in real space though if possible.} 
\item{The extra detail that's also needed here is to consider how we encode a 2-dimensional spatial process into our state vector, and how the elements of the resulting state vector might be correlated to one another depending on their spatial proximity. If we start with a Markovian Gaussian random field, we can derive the Mat\'{e}rn kernel over these spatial coordinates in order to correlate the state vectors in such a way.} 
\item{Also look into the Radial Basis Function (RBF) and higher-order derived kernels based on DALI expansion~\cite{sellentin2014breaking} in order to try and capture non-Gaussianity.}
\item{Spatial malaria models here:~\cite{smith2008towards,djordjevic2022stochastic}}
\item{Spatial helminth-like infections too would be a good template...}
\end{itemize}
}


\chapter{\sffamily Algo-trading on financial markets}

{\bfseries\sffamily Concept.} The idea here is to use the Q-Hawkes processes and the Bouchaud work to come up with some interesting simulations of financial markets. 

\textcolor{red}{
\begin{itemize}
\item{Fundamental simulation should be that of a proper limit order book microsim~\cite{bouchaud2018trades}}
\item{Algo trades using online learning of the market dynamics through Q-Hawkes processes} 
\end{itemize}
}



\chapter{\sffamily Learning from ants on curved surfaces}

{\bfseries\sffamily Concept.} The idea here is 

\section{\sffamily Diffusive limits for ant interactions}

\chapter{\sffamily Managing a Rugby match}

{\bfseries\sffamily Concept.} The idea here is 

\section{\sffamily Introduction}

Since the basic game engine will run using the \href{https://github.com/umbralcalc/stochadex}{stochadex} sampler, the novelties in this project are all in the design of the rugby match model itself. And, in this instance, we're not especially keen on spending a lot of time doing detailed data analysis to come up with the most realistic values for the parameters that are dreamed up here. Even though this would also be interesting.

One could do this data analysis, for instance, by scraping player-level performance data from one of the excellent websites that collect live commentary data such as \href{https://www.rugbypass.com/}{rugbypass.com} or \href{https://www.espn.co.uk/rugby/}{espn.co.uk/rugby}.

This game is primarily a way of testing out the interface of the stochadex for other users to build projects with. This should help to both iron out some of the kinks in the design, as well as prioritise adding some more convenience methods for event-based modelling into its code base.

\section{\sffamily Designing the event simulation engine}

We need to begin by specifying an appropriate event space to live in when simulating a rugby match. It is important at this level that events are defined in quite broadly applicable terms, as it will define the state space available to our stochastic sampler and hence the simulated game will never be allowed to exist outside of it. So, in order to capture the fully detailed range of events that are possible in a real-world match, we will need to be a little imaginative in how we define certain gameplay elements when we move through the space.

The diagrams below sum up what should hopefully work as a decent initial approximation while providing a little context with specific examples of play action.

\begin{figure}[h]
\includegraphics[width=8cm]{images/test.drawio.png}
\caption{Simplified event graph of a rugby union match - replace with drawio.}
\label{fig:event-graph}
\end{figure}

\begin{figure}[h]
\includegraphics[width=10cm]{images/test.drawio.png}
\caption{Optional model ideas - replace with drawio.}
\label{fig:model-ideas}
\end{figure}

\section{\sffamily Linking to player attributes}

\section{\sffamily Deciding on gameplay actions}

\section{\sffamily Writing the game itself}

\chapter{\sffamily Optimising relief chain logistics}

{\bfseries\sffamily Concept.} The idea here is 

\textcolor{red}{
\begin{itemize}
\item{Humanitarian aid logistics in response to flooding, fire or other natural disasters}
\item{Routing of transportation}
\item{Where to focus searches}
\item{Transportation size distribution}
\item{Supply chain logistics of resources and allocation of budget}
\item{Example paper here with stochastic network models~\cite{alem2016stochastic}}
\end{itemize}
}


%\appendix
%\chapter{First and only appendix}
\backmatter
\bibliographystyle{JHEP}
\bibliography{book}
\end{document}