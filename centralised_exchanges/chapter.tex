\chapter{\sffamily Centralised exchanges}

{\bfseries\sffamily Concept.} To define and develop an archetype simulation environment for centralised exchanges. In our classification scheme, this archetype is defined by an bidirectional star state partition graph topology and would make sense for simulations of financial, betting and housing markets as well as other forms of resource exchange. We will also discuss the typical ways in which the state partitions of the system may only partially be observed in realistic examples, and analyse how best to deal with each situation. For the mathematically-inclined, this chapter will define the mapping of our formalism to centralised exchanges. For the programmers, the software which is designed and described in this chapter can be found in the public Git respository here: \href{https://github.com/worldsoop/worldsoop}{https://github.com/worldsoop/worldsoop}.

\begin{figure}[h]
\centering
\includegraphics[width=9cm]{images/chapter-10-state-partition-graph.drawio.png}
\caption{State partition graph topology for centralised exchange archetypes.}
\label{fig:state-partition-graph-centralised-exchanges}
\end{figure}

\textcolor{red}{
\begin{itemize}
\item{Full sim: full event-based spatial stochastic model}
\item{Inference model: spatial mean field inference features using the probabilistic reweighting }
\item{Also use the likelihood-free inference model?}
\end{itemize}
}

\textcolor{red}{
\begin{itemize}
\item{Humanitarian aid logistics in response to flooding, fire or other natural disasters}
\item{Routing of transportation}
\item{Where to focus searches}
\item{Transportation size distribution}
\item{Supply chain logistics of resources and allocation of budget}
\item{Example paper here with stochastic network models~\cite{alem2016stochastic}}
\end{itemize}
}