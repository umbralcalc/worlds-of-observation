\chapter{\sffamily Generalised statistical inference}

{\bfseries\sffamily Concept.} The idea here is to extend the stochadex with tools for very generalised statistical inference of an explicit model that will work in nearly every situation where the user knows the model space. Probably need to exploit the phase space analogy of the formalism. For the mathematically-inclined, this chapter will... For the programmers, the software described in this chapter lives in the public Git repository: \href{https://github.com/umbralcalc/learnadex}{https://github.com/umbralcalc/learnadex}.


\section{\sffamily Methodology}

First there's the problem of knowing what the likelihood of the data is. The measurements $y, y', y'', \dots$ have either a known distribution which the user specifies or the filtering algorithm runs on this to get the data likelihood.

Once you have a data likelihood, then inference should proceed on the sampling domain side of things where the filtering algorithm is used again! This time with a tunable conditional probability that is a gaussian with mean and variance estimated directly from the history (don't optimise it like in Bayesian optimisation!) with a tunable exponential timestep kernel (timestep being the optimiser step here) and the resulting function can be optimised (or draw monte-carlo samples from) in an EM algorithm approach. The resulting Gaussian function could also exploit gradients for SGD.

Reference the contrast to ABC methods here, which involve approximating the data likelihood with a simple proximity function with a tolerance $\epsilon$. Also talk about the BOLFI method which does indeed use the full Bayesian optimisation as it goes.

